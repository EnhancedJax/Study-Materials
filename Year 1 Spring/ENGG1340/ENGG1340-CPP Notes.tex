\documentclass{article}

% ------------------------------------ %
%             Document Info            %
% ------------------------------------ %

\usepackage{../../LaTeX-Preamables/Clean}
\newcommand{\documentdate}{Spring 2024}
\newcommand{\documenttype}{C++ Notes for HKU}

% ------------------------------------ %
%              Title page              %
% ------------------------------------ %
\begin{document}
\begin{titlepage}
    \null\vfill % Add vertical space to center the title and author

    \centering
    \Huge\textbf{\documentname}

    \vspace{0.1cm}
    \Large\textbf{\documenttype\ $\cdot$ \documentdate}

    \vspace{1cm}
    \normalsize\textbf{Author:} \documentauthor

    \normalsize\textbf{Contact:} \documentauthorcontact

    MORE notes on my \href{https://jaxtam.dev/notes}{website}!
    \vfill % Add vertical space to center the remaining space
    \textcolor{gray}{Made for personal use only. Unmodified re-distribution is allowed. Content for reference only.}
\end{titlepage}

% ------------------------------------ %
%                  TOC                 %
% ------------------------------------ %

\newpage
\tableofcontents
\newpage
% ------------------------------------ %
%               Document               %
% ------------------------------------ %

\section{C++ basics}
Basic differences to Python:
\begin{enumerate}
    \item C++ is a \emph{compiled} language, which means that the code needs to be compiled before it can be run.
    \item C++ is a \emph{statically typed} language, which means that the type of a variable is known at compile time.
    \item C++ expressions end with a semicolon \Colorbox{superlightgray}{\lstinline|;|}.
\end{enumerate}
\subsection{Initialization}
For a \Colorbox{superlightgray}{\lstinline|.cpp|} file, we must include the following contents:
\lstinputlisting[language=c++, firstline=0, lastline=7]{code/cpp.cpp}
For this document, all codeblocks are assumed to be in the \Colorbox{superlightgray}{\lstinline|std|} namespace.

Without \Colorbox{superlightgray}{\lstinline|using namespace std;|}, we can use the \Colorbox{superlightgray}{\lstinline|std::|} prefix to access the objects and functions in the standard namespace.

\subsection{Types}
C++ is a statically typed language, which means that the type of a variable is known at compile time. The following are the basic types in C++:
\lstinputlisting[language=c++, firstline=24, lastline=30]{code/cpp.cpp}
\subsubsection{Chars}
We use single quotes to represent a \Colorbox{superlightgray}{\lstinline|char|}.
Note that there are no built-in string types in C++. A string is simple an \hyperref[sec:array]{array of characters}.

\Colorbox{superlightgray}{\lstinline|char|} stores a single character as a number, and the ASCII value of the character is stored in the \Colorbox{superlightgray}{\lstinline|char|} variable. If we assign a char to an interger, the ASCII value of the character is stored in the integer variable. This means that if we perform arithmetic operations on a \Colorbox{superlightgray}{\lstinline|char|}, the ASCII value of the character is used, and the result is stored as an integer.
\lstinputlisting[language=c++, firstline=206, lastline=209]{code/cpp.cpp}
\lstinputlisting[language=c++, firstline=210, lastline=216]{code/cpp.cpp}

\subsection{Variable declaration}
\lstinputlisting[language=c++, firstline=8, lastline=9]{code/cpp.cpp}

\subsection{Operators}
\lstinputlisting[language=c++, firstline=14, lastline=22]{code/cpp.cpp}

\subsection{cmath library}
The \Colorbox{superlightgray}{\lstinline|cmath|} library contains a set of functions that can be used to perform mathematical operations.
\lstinputlisting[language=c++, firstline=139, lastline=142]{code/cpp.cpp}

\subsection{cstdlib library}
The \Colorbox{superlightgray}{\lstinline|cstdlib|} library provides the \Colorbox{superlightgray}{\lstinline|rand()|} function, which returns a random number.
\lstinputlisting[language=c++, firstline=193, lastline=193]{code/cpp.cpp}

\section{Streams}
Streams are a sequence of bytes that represent a flow of data. In C++, we use the \Colorbox{superlightgray}{\lstinline|iostream|} library to work with streams. The \Colorbox{superlightgray}{\lstinline|iostream|} library contains the \Colorbox{superlightgray}{\lstinline|cin|} and \Colorbox{superlightgray}{\lstinline|cout|} objects, which are used to read and write data to the console, respectively.
\subsection{Character I/O Stream}
\lstinputlisting[language=c++, firstline=78, lastline=79]{code/cpp.cpp}
\begin{enumerate}
    \item \Colorbox{superlightgray}{\lstinline|>>|} is the \emph{extraction operator}
    \item \Colorbox{superlightgray}{\lstinline|<<|} is the \emph{insertion operator}
    \item We can use \Colorbox{superlightgray}{\lstinline|\|} to escape special characters
\end{enumerate}

\subsection{File I/O Stream}
We can also use the \Colorbox{superlightgray}{\lstinline|fstream|} library to work with files. The \Colorbox{superlightgray}{\lstinline|fstream|} library contains the \Colorbox{superlightgray}{\lstinline|ifstream|} and \Colorbox{superlightgray}{\lstinline|ofstream|} objects, which are used to read and write data to files, respectively.

\subsubsection{Reading from a file}
\lstinputlisting[language=c++, firstline=103, lastline=116]{code/cpp.cpp}

\subsubsection{Write to a file}
This following code creates \Colorbox{superlightgray}{\lstinline|output.txt|} if it doesn't exists and writes to it. If it does exist, it will be overwritten.
\lstinputlisting[language=c++, firstline=118, lastline=125]{code/cpp.cpp}

We can use the \Colorbox{superlightgray}{\lstinline|ios::app|} flag to append to the file instead of overwriting it.
\lstinputlisting[language=c++, firstline=127, lastline=127]{code/cpp.cpp}

\subsection{String Input Stream}
We can use the \Colorbox{superlightgray}{\lstinline|sstream|} library to work with strings. The \Colorbox{superlightgray}{\lstinline|sstream|} library contains the \Colorbox{superlightgray}{\lstinline|istringstream|}, which is used to read data from a string.

The following code separates the string \Colorbox{superlightgray}{\lstinline|str|} by spaces and stores the separated values in the \Colorbox{superlightgray}{\lstinline|word|} variable, then prints them line by line.

\lstinputlisting[language=c++, firstline=129, lastline=137]{code/cpp.cpp}

\subsection{Manipulators}
Manipulators are special functions that can be used to format the output of the \Colorbox{superlightgray}{\lstinline|cout|} object. They are presistent, meaning that they will affect all subsequent output until they are changed.

\subsubsection{End line}
The keyword \Colorbox{superlightgray}{\lstinline|endl|} is a manipulator that represents a new line. It is equivalent to the \Colorbox{superlightgray}{\lstinline|\n|} character.

\subsubsection{Decimal points}
We can fix the number of decimal points by using the \Colorbox{superlightgray}{\lstinline|fixed|} and \Colorbox{superlightgray}{\lstinline|setprecision|} functions from the \Colorbox{superlightgray}{\lstinline|iomanip|} library.
\lstinputlisting[language=c++, firstline=81, lastline=83]{code/cpp.cpp}

\subsubsection{Width and alignment}
We can fix the width of the output by using the \Colorbox{superlightgray}{\lstinline|setw|} function from the \Colorbox{superlightgray}{\lstinline|iomanip|} library.
\lstinputlisting[language=c++, firstline=99, lastline=101]{code/cpp.cpp}


\section{Control flow and functions}
\subsection{Basisc conditionals}
\lstinputlisting[language=c++, firstline=32, lastline=39]{code/cpp.cpp}
\subsection{Switches}
\lstinputlisting[language=c++, firstline=40, lastline=47]{code/cpp.cpp}
\subsection{Tentary operators}
\lstinputlisting[language=c++, firstline=49, lastline=49]{code/cpp.cpp}
\subsection{Loops}
\lstinputlisting[language=c++, firstline=51, lastline=63]{code/cpp.cpp}

\subsection{Functions}
\lstinputlisting[language=c++, firstline=65, lastline=68]{code/cpp.cpp}
Note that nested function declaration is not allowed.

\subsubsection{Pass by reference}
Pass-by-reference means to pass the \textbf{reference of a variable} as the parameter of a function. The called function can \textbf{modify the value of the variable} by using its reference passed in.

This example uses reference arguments:
\lstinputlisting[language=c++, firstline=85, lastline=94]{code/cpp.cpp}
This other example uses pointer arguments, so referenes are passed in:
\lstinputlisting[language=c++, firstline=272, lastline=280]{code/cpp.cpp}
More in \hyperref[sec:pointers]{reference operators}.

\label{sec:array}
\section{Arrays}
An array is a collection of elements of \textbf{the same type}. The elements of an array are stored in contiguous memory locations. C++ is a zero-indexed language, which means that the first element of an array is at index 0.

Arrays are static, so arrays cannot be variable-lengthed**. Additionally, only one element of an array can be changed at a time. \textbf{You cannot assign entire arrays to each other.}

\emph{** not part of the standard, but some compilers allow it.} Also check out \hyperref[sec:dynamicarrays]{dynamic arrays}.

\lstinputlisting[language=c++, firstline=70, lastline=73]{code/cpp.cpp}

\subsection{Function array parameters}
We can tell the function to expect an array as a parameter by using the square brackets \Colorbox{superlightgray}{\lstinline|[]|}.

Arrays passed into functions are \textbf{passed by reference}, so any changes made to the array in the function will be reflected in the original array.
\lstinputlisting[language=c++, firstline=195, lastline=201]{code/cpp.cpp}
For 2D arrays, we must specify the number of columns in the function parameter.
\lstinputlisting[language=c++, firstline=203, lastline=211]{code/cpp.cpp}

\subsection{Character arrays (C-string)}
A string is simply an array of characters. The last character of a string is the null character \Colorbox{superlightgray}{\lstinline|\0|}, which is used to denote the end of a string. The null character is automatically added to the end of a string when it is declared.
\lstinputlisting[language=c++, firstline=218, lastline=218]{code/cpp.cpp}
This also means that you cannot directly assign a string to the variable after declaration, as the same of arrays.
\lstinputlisting[language=c++, firstline=219, lastline=219]{code/cpp.cpp}

\subsection{The string class}
You can also use the string library to work with strings. The library provides the \Colorbox{superlightgray}{\lstinline|string|}class, as well as functions for working with strings.

\subsubsection{String concatenation}
We can use the \Colorbox{superlightgray}{\lstinline|+|} operator to concatenate strings. Note that at least one of the operands must be a string object.
\lstinputlisting[language=c++, firstline=221, lastline=225]{code/cpp.cpp}

\subsubsection{String comparison}
String comparisons are carried out character by character, and stops when a difference is found. The comparison is by ANSCII value.
\lstinputlisting[language=c++, firstline=227, lastline=233]{code/cpp.cpp}

\subsubsection{Member functions}
\lstinputlisting[language=c++, firstline=235, lastline=249]{code/cpp.cpp}

\subsection{Vectors}
Vectors are a sequence container that encapsulates dynamic size arrays.
\lstinputlisting[language=c++, firstline=178, lastline=189]{code/cpp.cpp}
We use vectors for dynamic arrays, and arrays for static arrays.

\section{Structures}
A structure is a user-defined data type that groups related variables of different types. A member is a variable within the structure.
\lstinputlisting[language=c++, firstline=151, lastline=153]{code/cpp.cpp}
Structure variables is declared and initialized similar to how you would for variables. We use the dot operator to access the members of a structure.
\lstinputlisting[language=c++, firstline=155, lastline=158]{code/cpp.cpp}
Note that we can only assign to a structure, and not perform operations on it.

Structures can be nested, and we can define member functions for a structure.
\lstinputlisting[language=c++, firstline=160, lastline=169]{code/cpp.cpp}

\section{Dynamic memory management}
\label{sec:pointers}
\subsection{Pointers}
A pointer is a variable that stores the \emph{memory address} of another variable.

\subsubsection{The reference operator}
\Colorbox{superlightgray}{\lstinline|&|} after the type name during delcaration indicates the variable serves as an alias to a variable (providing a \textbf{reference}).
\lstinputlisting[language=c++, firstline=257, lastline=260]{code/cpp.cpp}

\subsubsection{Pointer declaration and usage}
\Colorbox{superlightgray}{\lstinline|*|} after the type name during declaration indicates the variable is a pointer.

\Colorbox{superlightgray}{\lstinline|*|} before the variable name \textbf{dereferences} the pointer and gets the value stored in the memory address.

\Colorbox{superlightgray}{\lstinline|&|} before the variable name gives the \textbf{memory address} of the variable.

\lstinputlisting[language=c++, firstline=251, lastline=253]{code/cpp.cpp}

\subsubsection{Possible errors}
Note that pointers \textbf{must have the same type} as the variable they point to.
\lstinputlisting[language=c++, firstline=262, lastline=263]{code/cpp.cpp}
We must initialize a pointer to \Colorbox{superlightgray}{\lstinline|nullptr|} (same as 0) or a valid memory address before using it. If we don't, the pointer will be a \textbf{dangling pointer}.
\lstinputlisting[language=c++, firstline=255, lastline=255]{code/cpp.cpp}
Dereferencing a dangling pointer raises a \textbf{compile-time error}.

\subsubsection{Pointer operations}
We can carry out \emph{addition} and \emph{subtraction} on pointers, which will move the pointer to next or previous memory addresses. We can also carry out \emph{subtraction} on two pointers, which will give the number of elements between the two pointers.
\lstinputlisting[language=c++, firstline=296, lastline=299]{code/cpp.cpp}
We can also compare pointers with \Colorbox{superlightgray}{\lstinline|==|} and \Colorbox{superlightgray}{\lstinline|!=|}.


\subsubsection{Member access operator}
We can use the \Colorbox{superlightgray}{\lstinline|->|} operator to access the members of a structure that a pointer points to.
\lstinputlisting[language=c++, firstline=265, lastline=270]{code/cpp.cpp}

\subsubsection{Array pointers}
The name of an array is a \textbf{pointer to the first element} of the array. Therefore, we can use pointers to access the elements of an array as well.
\lstinputlisting[language=c++, firstline=282, lastline=285]{code/cpp.cpp}


\subsection{Dynamic variables and memory allocation}
Dynamic variables are variables that are allocated memory at runtime. We use the \Colorbox{superlightgray}{\lstinline|new|} operator to allocate memory for a dynamic variable, and the statement returns the memory address of the allocated memory. We can then store the address in a pointer to access the dynamic variable.
\lstinputlisting[language=c++, firstline=287, lastline=288]{code/cpp.cpp}
A dynamic variable is deallocated using the \Colorbox{superlightgray}{\lstinline|delete|} operator. It is important to deallocate memory when it is no longer needed to prevent memory leaks. Additionally, the pointer to the memory should be set to \Colorbox{superlightgray}{\lstinline|0|} after deallocation to prevent the pointer from becoming a \emph{dangling pointer}.
\lstinputlisting[language=c++, firstline=289, lastline=290]{code/cpp.cpp}

\label{sec:dynamicarrays}
\subsubsection{Dynamic arrays}
Unlike static arrays, dynamic arrays can have their size changed at runtime.
\lstinputlisting[language=c++, firstline=292, lastline=294]{code/cpp.cpp}

\subsection{Linked lists}
A linked list is a data structure that consists of a sequence of nodes (elements), where \textbf{each node points to the next node} in the sequence. The last node points to \Colorbox{superlightgray}{\lstinline|nullptr|}. Therefore, items \textbf{need not be stored in contiguous memory locations}.

This allows for \textbf{efficient insertion and deletion} of nodes, as we only need to change the pointers of the nodes around the element we want to insert or delete.

However, a linked list is a \textbf{sequential access} data structure, so to access the $n$th element, we must traverse the list from the first element. This is different from an array, where you can access the $n$th element directly.

\subsubsection{Implementation}
\lstinputlisting[language=c++, firstline=301, lastline=313]{code/cpp.cpp}
We tranverse through the linked list from \Colorbox{superlightgray}{\lstinline|head|} to access the data stored in the elements.
\lstinputlisting[language=c++, firstline=315, lastline=315]{code/cpp.cpp}
We can also use a while loop to do so:
\lstinputlisting[language=c++, firstline=317, lastline=321]{code/cpp.cpp}

\section{Recursion}
A recursion function can be defined by the following:
\begin{enumerate}
    \item $f(b)=k$ A basecase and it's result
    \item $f(a)=f(b)$ A recursive case which breaks the problem down into smaller problems for when it's not the basecase.
\end{enumerate}
We can implement a recursive function as followed:
\lstinputlisting[language=c++, firstline=338, lastline=344]{code/cpp.cpp}

\end{document}