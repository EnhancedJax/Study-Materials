\section{Data}

\hyperref[def:data]{Data} is the root of information systems.

\begin{definition}
    {DIKW Pyramid}
    \begin{itemize}
        \item \textbf{D}ata: raw facts
        \item \textbf{I}nformation: data with context
        \item \textbf{K}nowledge: information with understanding
        \item \textbf{W}isdom: knowledge with experience
    \end{itemize}
\end{definition}

\subsection{Personal information}

You expose a lot of personal information everytime you browse the web, including what you did, who you are, and where you are.

Websites use \textbf{cookies} to store information about you on your device, so that later when you visit their website again, they know it's you.

Even if you go incognito, your activity is still visible to your ISP and websites you visit.

\subsection{Business analytics}

\begin{definition}
    {Big data}
    A term that describes the \textit{massive} volume of data that companies collect. This data can be structured or unstructured, and are stored in distributed databases.

    These data are usually generated by users.

    The five Vs of big data are:
    \begin{itemize}
        \item Volume - Amounts of data
        \item Velocity - Speed of data generation
        \item Variety - Different types of data
        \item Veracity - Quality of data
        \item Value - Data is not useful unless it is used
    \end{itemize}

    The value of big data is how organizations uses them to generate \textbf{insights} from data analyses for \textbf{decision making}.
\end{definition}

Steps for conducting business analytics:
\begin{enumerate}
    \item Identify problems/objectives
    \item Prepare \textit{clean}, \textit{correct}, \textit{consistent} data, which should reflect \textbf{trends}
    \item Use \hyperref[analysis:technology]{technologies}, business knowledge and \hyperref[analysis:technique]{analysis techniques} to \textbf{generate insights} for better decision making
\end{enumerate}

\begin{theorem}
    {Analysis technologies}
    \begin{itemize}
        \item Collection
        \item Storage - Databases
        \item Visualization - Excel
        \item Processing - Excel
        \item Reporting - OLAP / Dashboards
    \end{itemize}
    Or other general purpose tools like Python
\end{theorem}
\label{analysis:technology}

\begin{theorem}
    {Analysis techniques}
    Analysis techniques can include the usage of data to:
    \begin{itemize}
        \item Describe history - Explore patterns and structures in data
        \item Explain history - Find relationships, correlation and causality
        \item Predict future - Trend / predictive analytics, ML
    \end{itemize}
\end{theorem}
\label{analysis:technique}

\begin{knBox}
    {Dashboards}
    Visual display of various \textbf{key performance indicators} and metrics, in a user friendly way. Might provide real-time data.
\end{knBox}

\begin{knBox}
    {OLAP}
    Online Analytical Processing, a technology that allows users to extract and view business data from different points of view.
\end{knBox}

There are some challenges in business analytics:

\begin{itemize}
    \item When a firm makes decisions based on wrong prediction, it is overexposed to risk
    \item Using bad data can give wrong estimates
    \item When the market does not behave as it did in the past, data-driven models are not effective
    \item Historical inconsistency
\end{itemize}
