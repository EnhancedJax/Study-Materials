\section{Business models}

A business model outlines how certain objectives are achieved, such as:

\begin{itemize}
    \item Value proposition
    \item Presence of business
    \item Marketing strategy
    \item Revenue strategy
    \item Pricing strategy
    \item Competitive environment
    \item Management team
\end{itemize}

\subsection{Presence of business}

We categorize the presence based on the level of physical / virtual presence.

\begin{itemize}
    \item Brick and mortar (offline)
    \item Click and Brick (hybrid)
    \item Click only (online)
\end{itemize}

\subsection{Revenue strategy}

Four common revenue strategies:

\begin{itemize}
    \item Subscription
    \item Transactional
    \item Advertising
    \item Traditional sales
    \item Affiliate marketing (profit from referrals)
\end{itemize}

\subsection{Pricing strategy}

Common pricing strategies in digital economy:
\begin{itemize}
    \item Freemium
          \begin{itemize}
              \item Combines "free" and "premium" features
              \item Basic features free, premium features paid
              \item Examples: YouTube, In-app purchases
          \end{itemize}
    \item Free trial
          \begin{itemize}
              \item Limited time test before purchase
              \item Common among software products
          \end{itemize}
\end{itemize}

\subsection{Value proposition}

\begin{itemize}
    \item Explains why customers should choose your business
    \item Must demonstrate greater value than competitors
    \item Key focus: market segments and product differentiation
\end{itemize}

Types of products:
\begin{itemize}
    \item Mainstream products
          \begin{itemize}
              \item Broad consumer appeal
              \item Serves mass market
              \item Examples: Ikea, ParknShop, Facebook, WhatsApp
          \end{itemize}
    \item Niche products
          \begin{itemize}
              \item Specialized appeal
              \item Serves specific market segments
              \item Examples: Designer furniture, International ParknShop, TikTok, Discord
          \end{itemize}
\end{itemize}

\subsection{Long tail strategy}

\begin{itemize}
    \item Profit from selling many low-volume, unique items rather than few high-volume items
    \item Particularly effective in e-commerce where:
          \begin{itemize}
              \item No physical inventory limitations
              \item Items easily discoverable online
              \item Helps compete in crowded markets
          \end{itemize}
\end{itemize}

\subsection{Product differentiation}

\begin{itemize}
    \item Vertical differentiation
          \begin{itemize}
              \item Universal preferences (better/faster is preferred)
              \item Examples: Quality, CPU speed, RAM, security, resolution
          \end{itemize}
    \item Horizontal differentiation
          \begin{itemize}
              \item Based on individual taste
              \item Examples: Color, style, flavor
          \end{itemize}
    \item Digital goods differentiation through:
          \begin{itemize}
              \item Security, resolution, genre
              \item Business presence, pricing, revenue models
          \end{itemize}
\end{itemize}

\subsection{Competition (Porter's Five Forces)}

\begin{enumerate}
    \item Rivalry among existing firms
          \begin{itemize}
              \item More competitors = less pricing power
              \item Example: Amazon sellers vs Amazon itself
          \end{itemize}

    \item Threat of new entrants
          \begin{itemize}
              \item Lower entry barriers = greater threat
              \item In digital economy, network effects often more important than costs
              \item Examples: Apps vs Semiconductor industry
          \end{itemize}

    \item Bargaining power of buyers
          \begin{itemize}
              \item Many small customers = more pricing power
              \item Examples: Content websites vs advertisers
          \end{itemize}

    \item Bargaining power of suppliers
          \begin{itemize}
              \item Fewer suppliers = more supplier power
              \item Examples: Content providers vs platforms
          \end{itemize}

    \item Threat of substitutes
          \begin{itemize}
              \item Available substitutes weaken market position
              \item Digital substitutes often less obvious
              \item Examples: Twitter vs Newspapers, Netflix vs theaters
          \end{itemize}
\end{enumerate}