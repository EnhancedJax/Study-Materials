\section{E-business models}

A business model outlines how certain objectives are achieved, such as:

\begin{itemize}
    \item Presence of business
    \item Revenue strategy
    \item Pricing strategy
    \item Value proposition
    \item Competition
\end{itemize}


\begin{knBox}
    {Presence of business}
    We categorize the presence based on the level of physical / virtual presence.

    \begin{itemize}
        \item Brick and mortar (offline)
        \item Click and Brick (hybrid)
        \item Click only (online)
    \end{itemize}
\end{knBox}

\begin{knBox}
    {Revenue strategy}
    Where does the revenue come from?

    \begin{itemize}
        \item Subscription
        \item Transactional
        \item Advertising
        \item Traditional sales
        \item Affiliate marketing (profit from referrals)
    \end{itemize}
\end{knBox}


\begin{knBox}
    {Pricing strategy}
    How are the products priced?

    \begin{itemize}
        \item Versioning - Multiple versions (e.g. professional, student)
        \item Bundling - Cheaper at bulk
        \item Freemium - Basic free, premium paid
        \item Free trial - Limited time test before purchase
    \end{itemize}
\end{knBox}

\begin{theorem}
    {Long tail strategy}
    \begin{itemize}
        \item Profit from selling many low-volume, unique items rather than few high-volume items
        \item Particularly effective in e-commerce where:
              \begin{itemize}
                  \item No physical inventory limitations
                  \item Items easily discoverable online
                  \item Helps compete in crowded markets
              \end{itemize}
    \end{itemize}
\end{theorem}

\begin{theorem}
    {Product differentiation}
    \begin{itemize}
        \item Vertical - By Standards
        \item Horizontal - By Tastes
    \end{itemize}
\end{theorem}

\begin{theorem}
    {Porter's Five Forces of Competition}
    \begin{enumerate}
        \item Rivalry among existing firms (no. of competitors $\uparrow$ $\implies$ competition $\uparrow$)
        \item Threat of new entrants (entry barriers $\uparrow$ $\implies$ competition $\downarrow$)
        \item Bargaining power of suppliers (no. of suppliers $\uparrow$ $\implies$ dependency $\downarrow$)
        \item Bargaining power of buyers (supply $\uparrow$ power of buyers $\uparrow$)
        \item Threat of substitutes (no. of alternatives $\uparrow$ $\implies$ competition $\uparrow$)
    \end{enumerate}
\end{theorem}
