\section{E-Commerce}

\begin{definition}
    {Types of e-commerce}
    Business, Customers
    \begin{itemize}
        \item B2C - (Nike)
        \item B2B - Typically on a \underline{wholesale} level (Apple buying batteries from Samsung)
        \item C2C - Carousell
        \item C2B - Freelancer, shutterstock
    \end{itemize}
\end{definition}

\begin{knBox}
    {O2O - Online to Offline / OMO - Online Merged Offline}
    Purchase for items or services online, but consume offline. Typically involves a brick-and-mortar store.

    \textit{Examples: Openrice, klook, uber}
\end{knBox}

\begin{table}[h!]
    \centering
    \begin{tabular}{|c|c|}
        \hline
        \textbf{Advantages}      & \textbf{Disadvantages}       \\
        \hline
        High accessibility       & Lack of physical experience  \\
        Convenience              & Delivery delays              \\
        Cost efficiency          & security concerns            \\
        Flexible business models & limited personal interaction \\
        \hline
    \end{tabular}
    \caption{Advantages and Disadvantages of E-commerce}
    \label{tab:econ_ad}
\end{table}

\subsection{Platform economy}
\label{subsec:platform}

Platforms are sensitive to the \hyperref[def:network]{\textbf{network effect}}. The more users on the platform, the more valuable it is. This allows the platform to have a significant advantage, even when innovative entrants have better products.

\begin{knBox}
    {Increase switching costs}
    Increasing switching costs will encourage users to stay on the platform. This can be done through:
    \begin{itemize}
        \item \textbf{Exclusivity}: Products cannot be accessed in other providers.
        \item \textbf{Sunk cost}: Rewards from time and effort spent on platform
        \item \textbf{Incompatibility}: Switching costs are high (e.g. learning, migration, etc.)
    \end{itemize}
\end{knBox}