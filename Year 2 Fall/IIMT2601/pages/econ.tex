\section{E-Commerce}

\begin{definition}
    {Types of e-commerce}
    Business, Customers
    \begin{itemize}
        \item B2C - (Nike)
        \item B2B - Typically on a \underline{wholesale} level (Apple buying batteries from Samsung)
        \item C2C - Carousell
        \item C2B - Freelancer, shutterstock
    \end{itemize}
\end{definition}

\begin{knBox}
    {O2O - Online to Offline}
    Purchase for items or services online, but consume offline. Typically involves a brick-and-mortar store.

    \textit{Examples: Openrice, klook, uber}
\end{knBox}

\subsection{Platform economy}
\label{subsec:platform}

Platforms are sensitive to the \hyperref[def:network]{\textbf{network effect}}. The more users on the platform, the more valuable it is. This allows the platform to have a significant advantage, even when innovative entrants have better products.

\begin{knBox}
    {Improve user retention}
    \begin{itemize}
        \item \textbf{Exclusivity}: Products cannot be accessed in other providers.
        \item \textbf{Sunk cost}: Rewards from time and effort spent on platform
        \item \textbf{Incompatibility}: Switching costs are high (e.g. learning, migration, etc.)
    \end{itemize}
\end{knBox}