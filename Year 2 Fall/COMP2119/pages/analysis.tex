\section{Algorithm Analysis}

We usually determine the efficiency of an algorithm by analyzing its time and space complexity.

\begin{definition}
    {Size of input}
    An \textbf{input} is a data structure that is given to an algorithm to solve a problem.
    The \textbf{size} of an input is the number $\mathbf{n}$ of elements in the input.
\end{definition}

\subsection{How do we assess time complexity?}

When we analyize time complexity, we are interested in the efficiency of an algorithm as a function of the size of the input.

As different machines have varying speeds of processors, time is not a good fit for measuring algorithm efficiency.

We can consider the number of operations an algorithm performs relative to the size of the input, which will isolate the algorithm from the performing machines.

However, different operations have varying efficiencies, so instead we consider the \textbf{growth rate} of the \textbf{total number of operations} as a function of the input size. The analysis of such is called \textbf{Asymptotic analysis}.


\begin{definition}
    {Time complexity}
    The \textbf{rate of growth} of the \textbf{total number of operations} an algorithm performs relative to \textbf{size of input}.
\end{definition}

\begin{definition}
    {Space complexity}
    The \textbf{rate of growth} of the \textbf{total amount of memory} an algorithm uses relative to \textbf{size of input}.
\end{definition}

\subsection{Asymptotic notation}

\begin{theorem}
    {Growth rate / Complexity: The basis of asymptotic notations}
    We are defining a certain function $\mathbf{T(n)}$ to represent the \textbf{total number of operations} with respect to $\mathbf{n}$ (the size of the input).

    When we say $T(n)$ is $A$ of $(g(n))$ (or $T(n)\in A(g(n))$), we are saying that the \textbf{growth rate} of $T(n)$ is bounded by $A(g(n))$ under a specific mathematical inequality defined by $A$.
\end{theorem}

\begin{definition}
    {General definition of asymptotic notations}
    For non-strict boundaries ($\mathcolorbox{pink}{\leq \geq}$)
    \[T(n) \in A(g(n))\ \text{iff}\ \mathcolorbox{yellow}{\exists\ c > 0}:\]
    \[T(n) \mathcolorbox{pink}{\simeq} c\cdot g(n)\ \forall\ n \geq n_0 > 0\]
    For strict boundaries: {$\mathcolorbox{pink}{< >}$}
    \[T(n) \in a(g(n))\ \text{if}\ \mathcolorbox{yellow}{\forall\ c > 0\ \exists\ n_0 \geq 0}:\]
    \[T(n) \mathcolorbox{pink}{\sim} c\cdot g(n)\ \forall\ n \geq n_0\]
    Note that the $\in$ is interchangable with $=$. The meaning does not really matter.
    \tcblower
    To prove non-strict boundaries, we need to show the existence of constants $c, n_0$ that satisfies the inequality.
    \label{def:asym}
    \ref{eg:asym_1} \ref{eg:asym_2}
\end{definition}

\begin{definition}
    {Definition of all asymptotic notations}
    \begin{tabular}{|c|c|c|c|}
        \hline
        \textbf{Notation} & \textbf{Condition}                                                                         & \textbf{Asymptotic boundary} & \textbf{Name}               \\ \hline
        $O(g)$            & $T(n) \mathcolorbox{pink}{\leq} c\cdot g(n)$                                             & Upper                        & Big O                       \\
        $\Omega(g)$       & $T(n) \mathcolorbox{pink}{\geq} c\cdot g(n)$                                             & Lower                        & Big Omega                   \\
        $\Theta(g)$       & $ c_1\cdot g(n)\mathcolorbox{pink}{\leq} T(n) \mathcolorbox{pink}{\leq} c_2\cdot g(n)$ & Tight                        & Big Theta / Order of growth \\
        $o(g)$            & $T(n) \mathcolorbox{pink}{<} c\cdot g(n)$                                                & Strictly upper               & Little o                    \\
        $\omega(g)$       & $T(n) \mathcolorbox{pink}{>} c\cdot g(n)$                                                & Strictly lower               & Little omega                \\ \hline
    \end{tabular}
\end{definition}

\begin{knBox}
    {Growth rate functions}
    We use the likeness of $g(n)$ to describe the complexity of $T(n)$. The following gives the common growth rate functions (increasing order of growth):

    \[
        1 < \sqrt{n} < \log n < n < n\log n < n^2 < n^3 < 2^n < n! < n^n
    \]

    Note that the higher the growth rate, the more complex the algorithm.

    Note the following growth rate equivilencies:
    \begin{itemize}
        \item $\log(n!)=\Theta(n\log(n))$ (no need to prove)
        \item $\Theta(\log_2n)=\Theta(\log_{10}n)$ as $\log_2n = \frac{\log n}{\log 2} = c\times\log n$
    \end{itemize}
\end{knBox}

\begin{theorem}
    {Identifying asymptotic growths}
    \textit{Non-strict}: Identify \textbf{the term with the highest growth rate} in $T(n)$ would be $g(n)$.

    \textit{Strict}: Choose the \textbf{next / previous} growth rate function with reference to the list depending on if it's upper / lower.

    You can use the same logic to determine if some $f(n)$ is $A/a$ of $g(n)$.

    Note that as the defintions are specified by an \textbf{inequality}, there could be \textbf{multiple satisfying} $g(n)$ for the same $T(n)$ and all of them are valid. However, keep in mind the only useful $g(n)$ for analysis would be closest to the condition boundary.

    If the function functuates, there is not a consistent growth rate unless a boundary is specified.
    \tcblower
    \label{thm:iasym}
    \ref{eg:iasym_1}
\end{theorem}

\subsection{Time complexity of code}

\begin{knBox}
    {Operation of time complexities}
    By definition:
    \[O(g)+O(g)=k\times O(g)=O(g)\]
    \[O(g)\times O(g)=O(g\times g)\]
    \[O(g_1)+O(g_2)=O(g_2),\quad g_2 > g_1\text{ in terms of order of growth}\]
\end{knBox}

Our goal is to examine how the runtime grows with respect to the input size. Consider the following example:

\begin{lstlisting}[language=Python]
                                        # Times ran O()
def find_max(arr):                      # 1
    if len(arr) == 0:                   # 1
        return None                     # 1
    max_val = arr[0]                    # 1
    for num in arr:                     # n
        if num > max_val:               # n
            max_val = num               # n
    return max_val                      # 1
\end{lstlisting}

It should be acknowledged that each operation / line would take different amounts of time to compute. However, as we are interested in the runtime $T(n)$'s growth rate, we can treat all operations to take $1$ unit of time.

Hence, $T(n)=O(1)+O(1)+O(1)+O(1)+O(n)+O(n)+O(n)+O(1)=O(n)$

\begin{theorem}
    {Typical growth rates and common occurance}
    \begin{tabular}{|l|p{0.7\textwidth}|}
        \hline
        \textbf{Growth} & \textbf{Occurance}                                                                                                                                        \\
        $1$             & Statments that run a set number of times with no respect to the input size                                                                                \\
        \hline
        $\log n$        & Algorithms that solve a big problem by transformation into a series of smaller problems, cutting the problem size by some constant fraction at each step. \\
        \hline
        $n$             & "for" loops that has $n$ amount of iterations per program run                                                                                             \\
        \hline
        $n\log n$       & Algorithms that solve a problem by breaking it up into smaller sub-problems, solving them independently, and then combining the solutions.                \\
        \hline
        $n^k$           & Nested "for" loops that has n amount of iterations per program run                                                                                        \\
        \hline
    \end{tabular}
\end{theorem}