\section{Analyzing financial statements}
\label{sec:analyze}

\subsection{Component percentages}
\label{sec:component-percentages}

Instead of using monetary amounts, we use \textbf{percentages} to compare the values to a \textbf{single base amount}. We divide all amounts on the statement by the base amount.

We choose the base amount based on the statement:
\begin{itemize}
    \item \textit{Income statement} $\rightarrow$ \textbf{Net sales}
    \item \textit{Balance sheet} $\rightarrow$ \textbf{Net assets}
\end{itemize}

\small
\begin{tcolorbox}[colframe=black,colback=white,title=Comparison of Income statement and component percentages]
    \begin{center}
        \textbf{XYZ Corporation}\\
        \textbf{Income statement}\\
        At December 31, 2025\\
        (In millions of dollars)
    \end{center}

    \begin{multicols}{3}
        \textbf{Operating revenue} \hfill \$1,000\\
        \textbf{Operating expenses} \\
        \begin{itemize}[label={}, leftmargin=*]
            \item Supplies Expense \hfill \$500
            \item Training Expenses \hfill \underline{300}
        \end{itemize}
        Total operating expenses \hfill \$800\\
        \textbf{Income from operations} \hfill \underline{\underline{\$200}}\\
        \vdots\\
        \textbf{Net income} \hfill \underline{\underline{\$200}}
        \columnbreak

        \begin{center}
            $\longrightarrow$
        \end{center}

        \columnbreak
        \textbf{Operating revenue} \hfill 100\%\\
        \textbf{Operating expenses} \\
        \begin{itemize}[label={}, leftmargin=*]
            \item Supplies Expense \hfill 50\%
            \item Training Expenses \hfill 30\%
        \end{itemize}
        Total operating expenses \hfill 80\%\\
        \textbf{Income from operations} \hfill 20\%\\
        \vdots\\
        \textbf{Net income} \hfill 20\%
    \end{multicols}

    \vspace{1em}

    \textit{\footnotesize{The notes are an integral part of these financial statements.}}
\end{tcolorbox}

\normalsize

\subsection{Ratio analysis}

\begin{itemize}
    \item \textbf{Ratios} are used to evaluate the financial performance of a company.
    \item They are used to compare the company's performance to:
          \begin{itemize}
              \item \textbf{Industry averages}
              \item \textbf{Historical data}
          \end{itemize}
          Without these benchmark values, no insights can be made.
    \item We can only compare ratios for two firms if they are \textbf{comparable} in terms of \textit{industry, operations and accounting policies}.
    \item A company's \textbf{accounting policies} influences its ratios.
\end{itemize}

\subsubsection{Profitability ratios}

Involves \textit{net income}.

\begin{table}[H]
    \centering
    \begin{tabular}{p{0.2\textwidth}p{0.4\textwidth}p{0.4\textwidth}}
        \hline
        \textbf{Ratio}           & \textbf{Formula}                & \textbf{Interpretation}                                                             \\
        \hline
        Net Profit Margin        & Net Income : Revenue            & Every \textbf{revenue dollar} generates $\underline{\hspace{1em}}$ \textbf{profit}. \\
        \hline
        Gross profit margin      & (Net income - COGS) : Revenue   & Every \textbf{revenue dollar} generates $\underline{\hspace{1em}}$ \textbf{profit}. \\
        \hline
        Return on Assets (ROA)   & Net Income : Total Assets       & Efficiency in using assets.                                                         \\
        \hline
        Return on Equity (ROE)   & Net Income : Average Equity     & Every \textbf{equity dollar} generates $\underline{\hspace{1em}}$ \textbf{profit}.  \\
        \hline
        Earnings per Share (EPS) & Net Income : Shares Outstanding & Amount of earnings attributable to each share of stock.                             \\
        \hline
        Quality of Income        & O Cash Flow : Net income        & Proportion of income in cash, ability to finance cash needs.                        \\
        \hline
    \end{tabular}
\end{table}

\subsubsection{Asset turnover ratios}

\begin{table}[H]
    \centering
    \begin{tabular}{p{0.2\textwidth}p{0.4\textwidth}p{0.4\textwidth}}
        \hline
        \textbf{Ratio}                 & \textbf{Formula}                          & \textbf{Interpretation}                                                               \\
        \hline
        Total Asset Turnover           & Operating revenue : Total Assets (sales:) & Every \textbf{asset dollar} generates $\underline{\hspace{1em}}$ \textbf{sales}.      \\
        \hline
        Fixed Asset Turnover           & Net Sales : Fixed Assets                  & Efficiency in using fixed assets.                                                     \\
        \hline
        Receivable Turnover            & Net Sales : Receivables                   & Every \textbf{receivable dollar} generates $\underline{\hspace{1em}}$ \textbf{sales}. \\
        \hline
        Inventory Turnover             & COGS : Inventory                          & Efficiency in managing inventory.                                                     \\
        \hline
        Average days to Sell Inventory & 365 : Inventory Turnover Ratio            & Number of days it takes to sell the inventory.                                        \\
        \hline
    \end{tabular}
\end{table}

Using ratios to analyze operating cycle

\subsubsection{Liquidity ratios}

Involves \textit{current assets} and \textit{current liabilities}.

\begin{table}[H]
    \centering
    \begin{tabular}{p{0.2\textwidth}p{0.4\textwidth}p{0.4\textwidth}}
        \hline
        \textbf{Ratio} & \textbf{Formula}                                   & \textbf{Interpretation}                                                                                              \\
        \hline
        Cash           & Cash Eqv. : Current Liabilities                    & Ability to pay off short-term debts without selling inventory.                                                       \\
        \hline
        Current        & Current Assets : Current Liabilities               & Ability to pay off short-term debts with current assets (\textbf{Short-term Liquidity}). Ideal range: \textbf{1.1-2} \\
        \hline
        Quick          & (Current Assets - Inventory) : Current Liabilities & Stricter liquidity ratio. Ideal range: \textbf{1-2}                                                                  \\
        \hline
    \end{tabular}
\end{table}

\subsubsection{Solvency ratios}

Involves \textit{debt}.

\begin{table}[H]
    \centering
    \begin{tabular}{p{0.2\textwidth}p{0.4\textwidth}p{0.4\textwidth}}
        \hline
        \textbf{Ratio}       & \textbf{Formula}                                              & \textbf{Interpretation}                        \\
        \hline
        Debt to Equity       & Total Liabilities : Total Equity                              & Amount of debt per dollar of equity.           \\
        \hline
        Time interest earned & NI + Interest Expense + Income tax Expense : Interest Expense & Ability to meet interest payments.             \\
        \hline
        Cash coverage        & O Cash Flow : Interest Expense                                & Cash from operations to meet interest payments \\
        \hline
    \end{tabular}
\end{table}

\subsubsection{Market ratios}

Involves \textbf{Market price} (per share \textbf{outstanding}).

\begin{table}[H]
    \centering
    \begin{tabular}{p{0.2\textwidth}p{0.4\textwidth}p{0.4\textwidth}}
        \hline
        \textbf{Ratio} & \textbf{Formula}                    & \textbf{Interpretation}              \\
        \hline
        Dividend yield & Dividend : Market price (per share) & Return on investment from dividends. \\
        \hline
        Price/Earnings & Market price : Earnings (per share) & Expected growth rate of earnings.    \\
        \hline
    \end{tabular}
\end{table}