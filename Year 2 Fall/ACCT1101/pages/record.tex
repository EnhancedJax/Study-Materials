\section{Bookkeeping}
\label{sec:record}

Bookkeeping is the process of recording transactions in a systematic way.

\begin{small}
    \textit{For convenience, in the notes "you" refers to "the company"}
\end{small}

\subsection{Identifying transactions}
\label{subsec:identify}

\begin{definition}
    {Accounts}
    An account stores all transactions related to a group of related items.
    \label{def:acc}
\end{definition}

\begin{theorem}
    {The 6 types of accounts}
    There are 6 types of accounts:
    \begin{itemize}
        \item \textbf{A}sset - Stuff you own
        \item \textbf{L}iability - Stuff you owe 3rd parties
        \item \textbf{E}quity (Stockholders' Equity) - Stuff for owners
              \begin{itemize}
                  \item \textbf{R}evenue - Money you earn from selling stuff
                  \item \textbf{E}xpense - Costs of doing business
                  \item \textbf{D}ividend - Money given to owners
              \end{itemize}
    \end{itemize}
    \label{thm:acc_types}
    R,E,D all feed into Equity. This will be further explored in \nameref{sec:financial_statements}.
\end{theorem}

\begin{knBox}
    {Current vs non-current A\&L}
    \begin{itemize}
        \item \textbf{Current A\&L} - Due within 1 year
        \item \textbf{Non-current A\&L} - Stuff that stays with you for more than 1 year
    \end{itemize}
\end{knBox}

\begin{knBox}
    {Operating vs non-operating R\&E}
    \begin{itemize}
        \item \textbf{Operating} - Central focus of business
        \item \textbf{Non-operating} - Not related to the main business
    \end{itemize}
\end{knBox}

\begin{knBox}
    {Contra accounts}
    Contra accounts are accounts that have a normal balance \textbf{opposite} to the account type. The following is how we present them in a financial statement:

    \tcblower

    \begin{tabular}{ll}
        \textbf{Assets}                             &       \\
        Accounts receivable                         & \$10  \\
        \quad Less: Allowance for doubtful accounts & (\$1) \\
    \end{tabular}
\end{knBox}
\label{def:contra}

\subsubsection{Categorizing transactions}

\begin{table}[H]
    \centering
    \begin{tabular}{|p{0.1\textwidth}|p{0.35\textwidth}|p{0.35\textwidth}|}
        \hline
                    & \textbf{A}                       & \textbf{L}                                 \\
        \hline
        \textbf{C}  & Cash                             & Accounts Payable                           \\
                    & Short-Term Investments           & Accrued Expenses Payable                   \\
                    & Accounts Receivable              & Notes Payable                              \\
                    & Notes Receivable                 & Taxes Payable                              \\
                    & Inventory (to be sold)           & Unearned Revenue                           \\
                    & Supplies                         & Bonds Payable (short-term)                 \\
                    & Prepaid Expenses                 & Short term debts                           \\
                    &                                  & Dividends payable                          \\
        \hline
        \textbf{NC} & Long-Term Investments            & Bonds Payable (long-term)                  \\
                    & Equipment                        & Long term debts                            \\
                    & Buildings                        &                                            \\
                    & Land                             &                                            \\
                    & Intangibles                      &                                            \\
        \hline
        Keywords    & prepaid expense, accrued revenue & payable, prepaid revenue, accrued expenses \\
        \hline
    \end{tabular}
    \caption{Classification of typical account titles (A,L)}
\end{table}

\begin{table}[H]
    \centering
    \begin{tabular}{|p{0.9\textwidth}|}
        \hline
        \textbf{Equity}                 \\
        \hline
        Common Stock \textbf{(CS)}      \\
        Retained Earnings \textbf{(RE)} \\
        Additional Paid-in Capital      \\
        Treasury Stock                  \\
        \hline
    \end{tabular}
    \caption{Common Equity accounts}
\end{table}

\hyperref[subsec:stockholder]{Learn more} about Stockholder's Equity.

\begin{table}[H]
    \centering
    \begin{tabular}{|l|p{0.3\textwidth}|p{0.3\textwidth}|}
        \hline
                    & \textbf{Revenue}       & \textbf{Expenses}         \\
        \hline
        \textbf{O}  & Sales Revenue          & Cost of Goods Sold (COGS) \\
                    & Service Revenue        & Depreciation expense      \\
                    &                        & Wages expense             \\
                    &                        & Rent expense              \\
                    &                        & Insurance expense         \\
                    &                        & Repairs expense           \\
        \hline
        \textbf{NO} & Interest Revenue       & Interest Expense          \\
                    & Dividend Revenue       & Tax Expense               \\
                    & Gain on Sale of Assets &                           \\
                    & Tax Revenue            &                           \\
        \hline
    \end{tabular}
    \caption{Common Revenue and Expense accounts categorized by operation}
\end{table}

\begin{knBox}
    {Cash and cash equivalents}
    The most liquid assets that can be \textbf{quickly converted to cash}. We usually group these into a single \textit{cash} account.
\end{knBox}

\begin{knBox}
    {Notes Receivable / Payable}
    The money you own / owe with an interest rate.
    \tcblower
    \textit{Example 1: } You borrowed \$1000 from a bank. (Payable)
\end{knBox}
\label{def:nrp}

\begin{knBox}
    {Debts}
    The money you borrow from a bank.
\end{knBox}

\subsection{Preparing journal entries}
\label{subsec:prepare}
\begin{definition}
    {The accounting equation}
    \[A = L + E\]
    This equation is based on the principle that \textbf{Stuff that you OWN = Stuff that you OWE}.
    \label{def:account_equation}
\end{definition}

\begin{theorem}
    {Double-entry bookkeeping}
    From money to go to one account \textit{(source)}, it must come from another \textit{(destination)}. This is the \textit{duality} of every transaction in accounting.

    $\uparrow$ \textbf{Debit (Dr)} is the term to describe the \textbf{destination} of money flow, whereas $\downarrow$ \textbf{Credit (Cr)} is the term to describe the \textbf{source} of money flow.
\end{theorem}
\label{thm:debit_credit}

\begin{knBox}
    {Categorizing transactions as Dr / Cr}
    We can categorize transactions as debits or credits by their accounts' type:
    \vspace{0.5em}
    \begin{center}
        \small
        Dr - Cr
        \normalsize

        \textbf{DEALER}

        Dividend, Expense, Asset - Liability, Equity, Revenue
    \end{center}
    They would be a Dr / Cr if their value was positive. Otherwise, they would be the other way around.

    \textit{This is based on rearranging the accounting equation and expanding its terms.}
\end{knBox}

\begin{definition}
    {Journal entry}
    A record of a transaction in the journal \underline{during the accounting period}. The following is the format:
    \begin{itemize}
        \item Date
        \item Account
        \item Description
    \end{itemize}
    Amounts credited are usually \textbf{indented}.
\end{definition}

\small
\begin{tcolorbox}[colframe=black,colback=white,title=Example Journal Entry]
    \begin{tabular}{llll}
        \textbf{Date} & \textbf{Account}                                                               & \textbf{Dr} & \textbf{Cr} \\
        \hline
                      & \textit{in thousands}                                                          &             &             \\
        Jan 1, 2023   & Cash                                                                           & \$1         &             \\
                      & \quad Common Stock                                                             &             & \$1         \\
                      & \multicolumn{3}{l}{\textit{(To record the issuance of common stock for cash)}}                             \\
    \end{tabular}
\end{tcolorbox}
\normalsize

\subsection{Posting to general ledger}
\label{subsec:post}

\begin{knBox}
    {T-accounts}
    T-account is a way to visualize an account. \textbf{Dr} goes on the left, \textbf{Cr} goes on the right. The closing balance is displayed on the corresponding side.
    \tcblower
    \taccount{Cash}
    {\$1000\\\hline\underline{\underline{\$1000}}}
    {}
    \taccount{Common Stock}
    {\\}
    {\$1000\\\hline\underline{\underline{\$1000}}}
\end{knBox}

\begin{definition}
    {General ledger}
    A general ledger is a collection / database of accounts that store all transactions for you. Each account has their own general ledger.
    \label{def:gl}
\end{definition}

\small
\begin{tcolorbox}[colframe=black,colback=white,title=Example General Ledger]
    \textbf{Cash} \\
    \begin{tabular}{llrrr}
        Date  & Description       & Debit   & Credit & Balance \\
        \hline
        Jan 1 & Issuance of stock & \$1,000 &        & \$1,000 \\
        \hline
              & Ending Balance    &         &        & \$1,000 \\
    \end{tabular}

    \vspace{0.5em}
    \textbf{Common Stock}\\
    \begin{tabular}{llrrr}
        Date  & Description       & Debit & Credit  & Balance \\
        \hline
        Jan 1 & Issuance of stock &       & \$1,000 & \$1,000 \\
        \hline
              & Ending Balance    &       &         & \$1,000 \\
    \end{tabular}
\end{tcolorbox}
\normalsize

\subsection{Reporting and intepreting Stockholder's Equity}
\label{subsec:stockholder}

Corporate businesses has an advantage over proprietorships of the ease of participation in ownership. This is done through the issuance of \textbf{common stock}. Owners of common stock are known as \textit{stockholders / shareholders}.

\textbf{Dividends} are the money paid to the shareholders from the company's profits, when the company decides to distribute them.

\subsubsection{Benefits of stock ownership}

Stockholders have the following benefits:

\begin{itemize}
    \item Voice in management
    \item Dividends
    \item Claim of assets upon liquidation of company
\end{itemize}

\subsubsection{Authorized, issue and outstanding shares}

The following are the categorization of stocks:

\begin{itemize}
    \item Authorized - Max. number that can be sold
          \begin{itemize}
              \item Issued - Sold to the public
                    \begin{itemize}
                        \item \colorbox{pink}{\textbf{Common stock}} - Currently held by the public (Outstanding stock)
                        \item \colorbox{pink}{\textbf{Treasury stock}} - Bought back by the company from the public
                        \item \hyperref[def:preferred_stock]{Preferred stock}
                    \end{itemize}
              \item Unissued - Never been sold
          \end{itemize}
\end{itemize}

\begin{definition}
    {Common stock}
    The shares currently held by public.
\end{definition}

\begin{definition}
    {Treasury stock}
    The shares bought back by the company from the public. This is a \hyperref[def:contra]{contra}-equity account, as can be thought of as being opposite to common stock.
\end{definition}

\begin{knBox}
    {Par vs market value}
    \begin{itemize}
        \item \textbf{Par value} - The minimum price of the stock
        \item \textbf{Market value} - The price of the stock in the market
    \end{itemize}
\end{knBox}
\label{def:par_market}

\begin{theorem}
    {Preferred stock}
    A special category of stocks. They have a \textit{fixed dividend rate} which is \textit{paid before common stockholders}. However, they do not have voting rights.

    The preferred stock holders offers two \textbf{dividend options}:
    \begin{enumerate}
        \item \textbf{Current} - Paid in the current period, limits to before common stockholders
        \item \textbf{Cummulative} - Paid in the future if not paid in the current period. This amount is called \textbf{dividends in arrears}.
    \end{enumerate}
    Preferred stock has a \textbf{fixed dividend rate}, which is given by:
    \[\text{\textbf{P}referred \textbf{D}ividends} = \text{Par value} \times \text{Dividend rate} \times \text{Number of shares}\]
    They have a separate par value from common stock.

    As their dividends are paid first:
    \[\text{\textbf{C}ommon \textbf{D}ividends} = D - PD\]

    \tcblower

    "Preferred stock: \$20 par value, 6\% rate, 1000 shares. Common stock: \$10 par value, 5000 shares." $\implies$ PD = $20 \times 1000 \times 0.1 = 2000$

    Consider a issurance of \$3000 dividends, for \textit{current} preference:

    $\text{PD}=2000$,

    $\text{CD} = 3000 - \text{PD} = 1000$

    Consider a issurance of \$10000 dividends, for \textit{cummulative} preference, dividends in arrears for 2 years:

    $\text{PD} = 2000 + 2000 * 2 = 4000$ (first 2000 is the current year's),

    $\text{CD} = 10000 - \text{PD} = 6000$


\end{theorem}
\label{def:preferred_stock}

\subsubsection{Reporting stockholder's equity}

\begin{theorem}
    {Sale of stock}
    When a stock is sold. Other than recording the stock issurance, we also need to record the \colorbox{pink}{\textbf{additional paid-in capital}} (APC) when the stock is sold for more than the \hyperref[def:par_market]{par value}.

    \begin{center}
        $\Delta$\textbf{C}ommon \textbf{S}tock = Stock Issurance = Number of shares $\times$ Par value

        $\Delta$\textbf{APC} = Value - Stock Issurance
    \end{center}
    \tcblower
    "Issued 200m additional shares of \$0.01 par value for \$16m" $\implies\ \Delta$APC = 14m, $\Delta$CS = 2m

    \vspace{1em}

    \textit{Example entry:}
    \begin{tabular}{|lll|}
        \hline
        Cash                             & \$16 &    \\
        \quad Common Stock               &      & 2  \\
        \quad Additional Paid-in Capital &      & 14 \\
        \hline
    \end{tabular}
    \label{def:sale_of_stock}
\end{theorem}

\begin{theorem}
    {Declaring dividends}
    When we declare a dividend, we \textbf{allocate money} to the owners from retained earnings.
    \tcblower
    "Declared dividends of \$2m"

    \vspace{1em}
    \textit{Example entry:}
    \begin{tabular}{|lll|}
        \hline
        Retained earnings       & \$2 &   \\
        \quad Dividends payable &     & 2 \\
        \hline
    \end{tabular}
\end{theorem}

\begin{theorem}
    {Repurchase and reissurance of stock}
    When we \textit{re}issure stock, we need to credit treasury stock instead of common stock, as we pull from our own stock. We also need to record the \hyperref[def:sale_of_stock]{additional paid-in capital} when we sell the stock for more than its par value.
    \tcblower
    "Reacquired 10 shares of common stock when selling for \$14 per share." $\implies \Delta$Treasury = 140

    \vspace{1em}

    \textit{Example entry:}
    \begin{tabular}{|lll|}
        \hline
        Treasury stock & \$140 &     \\
        \quad Cash     &       & 140 \\
        \hline
    \end{tabular}

    \vspace{2em}

    "Reissued 1 shares of treasury stock at \$13 per share" $\implies \Delta$APC = -1 (less than par value), $\Delta$Treasury = 14

    \vspace{1em}

    \textit{Example entry:}
    \begin{tabular}{|lll|}
        \hline
        Cash                       & \$13 &    \\
        Additional paid-in capital & 1    &    \\
        \quad Treasury             &      & 14 \\
        \hline
    \end{tabular}
\end{theorem}

These concepts make up the \hyperref[sec:statement_se]{statement of stockholder's equity}.

\subsection{Reporting inventories \& COGS}

\begin{definition}
    {Inventory}
    The goods that you have for sale as a retailer, or the raw materials that you have for sale as a manufacturer.
\end{definition}

\begin{knBox}
    {Supplies vs. Inventory}
    Supplies are consumables (such as paper, boxes etc.) for the business.
\end{knBox}

As a retailer, you purchase the goods to sell as inventory from a supplier, and make a sale to a customer. We need to record this purchase, sale, COGS and the change in inventory.

\begin{theorem}
    {Reporting inventories and COGS (Inventory system)}
    There are two ways we can record COGS and inventory in an according period:
    \begin{itemize}
        \item \textbf{As the goods are sold / purchased} (Perpetual)
              \begin{itemize}
                  \item[] $\checkmark$ Updates accounts in real-time
                  \item[] $\times$ More complex, more costly
              \end{itemize}
        \item \textbf{At the end of the period} (Periodic)
              \begin{itemize}
                  \item[] $\times$ Does not update accounts in real-time
                  \item[] $\checkmark$ Less complex, less costly
              \end{itemize}
    \end{itemize}
\end{theorem}


\small
\begin{tcolorbox}[colframe=black,colback=white,title=Example to compare the two inventory systems]
    \begin{minipage}[t]{0.48\textwidth}
        \textbf{Perpetual System:}

        \vspace{1.3em}

        When we restock our inventory:

        \begin{tabular}{llll}
            \textbf{Date} & \textbf{Account}           & \textbf{Dr} & \textbf{Cr} \\
            \hline
            Jan 1, 2023   & \colorbox{lime}{Inventory} & \$1         &             \\
                          & \quad {Cash}               &             & \$1         \\
        \end{tabular}

        \vspace{1.3em}

        When we make a sale:

        \begin{tabular}{llll}
            \textbf{Date} & \textbf{Account}                                                 & \textbf{Dr}            & \textbf{Cr} \\
            \hline
            Jan 2, 2023   & Cash                                                             & \$2                    &             \\
                          & \quad Sales Revenue                                              &                        & \$2         \\
                          & \multicolumn{3}{l}{\textit{(To record the sale of goods)}}                                              \\
            Jan 2, 2023   & Cost of Goods Sold                                               & \colorbox{yellow}{\$1} &             \\
                          & \quad Inventory                                                  &                        & \$1         \\
                          & \multicolumn{3}{l}{\textit{(\colorbox{lime}{Update real-time})}}                                        \\
        \end{tabular}
    \end{minipage}
    \hfill
    \begin{minipage}[t]{0.48\textwidth}
        \textbf{Periodic System:}

        \vspace{1.3em}

        When we restock our inventory:

        \begin{tabular}{llll}
            \textbf{Date} & \textbf{Account}           & \textbf{Dr} & \textbf{Cr} \\
            \hline
            Jan 1, 2023   & \colorbox{lime}{Purchases} & \$1         &             \\
                          & \quad Cash                 &             & \$1         \\
        \end{tabular}

        \vspace{1.3em}

        When we make a sale:

        \begin{tabular}{llll}
            \textbf{Date} & \textbf{Account}                                           & \textbf{Dr} & \textbf{Cr} \\
            \hline
            Jan 2, 2023   & Cash                                                       & \$2         &             \\
                          & \quad Sales Revenue                                        &             & \$2         \\
                          & \multicolumn{3}{l}{\textit{(To record the sale of goods)}}                             \\
        \end{tabular}

        \vspace{1.3em}

        At the end of the period:

        \begin{tabular}{llll}
            \textbf{Date} & \textbf{Account}                                                               & \textbf{Dr}            & \textbf{Cr} \\
            \hline
            Dec 31, 2023  & Inventory                                                                      & \$1                    &             \\
                          & \quad Purchases                                                                &                        & \$1         \\
                          & \multicolumn{3}{l}{\textit{(\colorbox{lime}{Update inventory restock first})}}                                        \\
                          & Cost of Goods Sold                                                             & \colorbox{yellow}{\$1} &             \\
                          & \quad Inventory                                                                &                        & \$1         \\
                          & \multicolumn{3}{l}{\textit{(Update COGS after)}}                                                                      \\
        \end{tabular}
    \end{minipage}
\end{tcolorbox}
\normalsize

The difference between the COGS and the revenue will reflect the profit accordingly when we produce the \hyperref[sec:income_statement]{income statement}.

The above example assumes that the \colorbox{yellow}{cost of inventory} is the same for each unit on every restock, which is not always the case.

\begin{theorem}
    {Inventory cost flow table}
    Reflects the changes in inventory during an accounting period, with the total cost of good sold depending on \hyperref[thm:cost_flow]{cost flow assumptions}.
\end{theorem}

\small
\begin{tcolorbox}[colframe=black,colback=white,title=Example of inventory cost flow table]
    During the period, the following transactions occurred:
    \begin{enumerate}
        \item 1/1: First purchase of goods 100 units at \$1 each
        \item 1/2: Restocked 200 units by \$400
        \item 1/3: Sold 150 units for \$450
    \end{enumerate}

    \vspace{1em}

    \begin{tabular}{l|l|c|c|c}
        \textbf{Date} & \textbf{Action}          & \textbf{Quantity} & \textbf{Unit cost} & \textbf{Total cost}                               \\
        \hline
        Jan 1, 2023   & Opening                  & 100               & \$1                & \$100                                             \\
        Jan 2, 2023   & Purchase                 & 200               & 2                  & 400                                               \\
        \hline
                      & Goods available for sale & 300               & /                  & 500                                               \\
        Jan 3, 2023   & Sale                     & (150)             & /                  & (\colorbox{yellow}{?}) $\leftarrow$ \textit{COGS} \\
        \hline
                      & Ending inventory         & 150               & /                  & $500-?$                                           \\
    \end{tabular}

    \vspace{1em}

    \textit{\colorbox{yellow}{?} COGS depends on the \hyperref[thm:cost_flow]{cost flow assumptions}.}
\end{tcolorbox}
\normalsize

\begin{theorem}
    {Cost flow assumptions}
    To determine the COGS when we purchased the goods at different costs, we can use the following assumptions:
    \begin{itemize}
        \item \textbf{FIFO} - First In, First Out
        \item \textbf{LIFO} - Last In, First Out
        \item \textbf{AVCO} - Average cost of all units (Total cost / Available units)
        \item \textbf{Specific Identification} - Ratio / quantity specified
    \end{itemize}
    They are assumptions because we do not know which physical units were sold.
    \tcblower
    \begin{enumerate}
        \item \colorbox{yellow}{?}$\ =1(100)+2(50)=200$
        \item \colorbox{yellow}{?}$\ =2(100)+1(50)=250$
        \item \colorbox{yellow}{?}$\ =\frac{500}{300}(150)=250$
    \end{enumerate}
\end{theorem}
\label{thm:cost_flow}

\begin{knBox}
    {Which assumption is better?}
    We must take into account for the \textbf{rise / drop in inventory prices during the period} when considering which assumption is better.

    In a period of \textbf{rising prices}:
    \begin{itemize}
        \item FIFO - Higher net income (less COGS expenses)
        \item LIFO - Lower net income (more COGS expenses)
    \end{itemize}
    \textit{Note: Accounting rules require companies to apply their accounting methods on a consistent basis over time. (Can't change frequently)}
\end{knBox}

\subsubsection{Net realizable value of inventory}

\begin{definition}
    {Net realizable value}
    The actual of the inventory that the company can profit from.
    \[NRV = \text{Selling price} - \text{Cost to sell}\]
\end{definition}

\begin{theorem}
    {Reporting Inventory at Lower of Cost or Net Realizable Value}
    We need to report inventory at its \textbf{minimum} of the \textit{cost} or NRV. This is to prevent \textbf{overstating} the value of inventory.

    If $NRV < Cost$, the company would make a \textbf{“write-down” entry} of $(Cost - NRV) \times units$ to reduce the inventory balance to NRV. (As they are originally recorded at cost).
    \tcblower
    \textit{Example: The company has 1000 units of inventory at \$10 each. The NRV is \$8.}

    The valuation used at the end of this period would be $1000 \times 8 = 8000$. As $NRV < Cost$, we need to make the following entry:

    \begin{tabular}{llll}
        \textbf{Account}  & \textbf{Dr} & \textbf{Cr} \\
        \hline
        COGS              & 2000        &             \\
        $\quad$ Inventory &             & 2000        \\
    \end{tabular}

    $(10-8)\times1000=2000$
\end{theorem}

\subsection{Reporting equipments}

\begin{definition}
    {Acquisition costs}
    We include the following \textbf{acquisition costs} in the value of an equipment upon acquisition:
    \begin{itemize}
        \item Purchase price
        \item Sales taxes
        \item Legal fees
        \item Transportation costs
        \item Installation and preparation costs
    \end{itemize}
    The act of inclusion is also called \textbf{capitalization}.
    \tcblower
    \textit{Example: Soutwest Airlines purchased aircraft of \$20, and paid \$1 for transportation and \$2 for installation.}

    \begin{tabular}{llll}
        \textbf{Account} & \textbf{Dr} & \textbf{Cr} \\
        \hline
        Equipment        & \$23        &             \\
        $\quad$ Cash     &             & \$23        \\
    \end{tabular}
\end{definition}

\begin{theorem}
    {Reporting maintenance and improvements of equipment}
    \begin{itemize}
        \item \textbf{Maintenance} - Expense
        \item \textbf{Improvements} - Capitalized
    \end{itemize}
    Some identifying characteristics of improvements are \textit{large sums of money}, \textit{longer useful life}, and \textit{increased efficiency}.
\end{theorem}

\begin{theorem}
    {Reporting asset disposal}
    \begin{enumerate}
        \item Record \textbf{\hyperref[subsec:depreciation]{depreciation} expense} of the item for this period
        \item Calculate and write-off \textbf{accumulated depreciation} for this item
        \item Record the disposal of the asset by crediting it
        \item Record "gain on sale of assets" (revenue) or "loss on sale of assets" (expense) by balancing the entry
    \end{enumerate}
    \tcblower
    \textit{Example: Southwest Airlines sold flight equipment for \$11 cash at the end of its 17th year of use. The flight equipment originally cost \$30 and was depreciated using the straight-line method with zero residual value and a useful life of 25 years.}

    \begin{tabular}{llll}
           & \textbf{Account}                 & \textbf{Dr} & \textbf{Cr} \\
        \hline
        1. & Depreciation expense             & \$1.2       &             \\
           & $\quad$ Accumulated depreciation &             & \$1.2       \\
           & (${30}/{25}=1.2$)                &             &             \\
        \hline
        2. & Cash                             & \$11        &             \\
           & Accumulated depreciation         & 20.4        &             \\
           & $\quad$ Equipment                &             & 30          \\
           & $\quad$ Gain on sale of assets   &             & 1.4         \\
           & ($1.2 \times 17=20.4$)           &             &             \\
    \end{tabular}
\end{theorem}
\label{subsec:asset_disposal}

\begin{knBox}
    {Change in estimates}
    If we change the estimates of the useful life or residual value of an asset, we need to \textbf{recalculate the net book value} of the asset.

    \tcblower

    \textit{Example: Southwest Airlines spent \$200 improving the flight equipment of \$1000, changing the useful life of the flight equipment from 25 years to 30 years, with 0 residual value. It was in service for 2 years. Find the new net book value and calculate depreciation for this period.}

    Computation for change in estimates:
    \begin{itemize}
        \item Original cost: \$1000
        \item Less: Accumulated depreciation: $\frac{1000}{25}\times2=\$80$
        \item Improvement capitalized cost: \$200
        \item \textbf{New cost}: \$1120
    \end{itemize}
    Depreciation expense: $\frac{1120}{30-2}=\$40$
\end{knBox}

\subsection{Reporting taxes}

\begin{definition}
    {Income tax}
    The tax that you pay on your income. It is calculated as a percentage of your income. We usually record it as \textbf{tax payable} until we actually pay the tax.
    \tcblower
    \textit{Example: Southwest Airlines has a tax rate of 30\%. They earned \$1000 in revenue. Record the tax expense.}

    \begin{tabular}{llll}
        \textbf{Account}           & \textbf{Dr} & \textbf{Cr} \\
        \hline
        Income tax expense         & \$300       &             \\
        $\quad$ Income tax payable &             & \$300       \\
    \end{tabular}
\end{definition}