\section{Closing Entries}
\label{sec:closing_entries}

\begin{definition}
    {Categorizing accounts as Temporary / Permanent}
    A permanent account carries balance forward to the next period, while a temporary account resets to zero at the end of the period.

    \vspace{0.5em}
    \begin{center}
        \small
        Temp - Perm
        \normalsize

        \textbf{RED ALE}

        Revenue, Expense, Dividend - Asset, Liability, Equity
    \end{center}
    \label{def:temp_perm}
    \tcblower
    \textit{Related: } \hyperref[thm:adjusting_entry]{Adjusting entry}
\end{definition}

\begin{definition}
    {Closing entries}
    Entry to reset temporary accounts to zero by transferring their balances to \textbf{Retained Earnings}.

    The entry is created after the \nameref{sec:income_statement}, and before the \nameref{sec:balance_sheet}, as otherwise there will be nothing to display.
\end{definition}

\begin{tcolorbox}[colframe=black,colback=white,title=Example Closing Entry]
    Considering the following adjusted trial balance at the end of the period:

    \vspace{1em}

    \begin{tabular}{lrr}
        \textbf{Account} & \textbf{Debit} & \textbf{Credit} \\
        \hline
        Cash             & \$5            &                 \\
        Revenue          &                & \$15            \\
        Expense          & \$5            &                 \\
        Dividends        & \$5            &                 \\
        \hline
        \textbf{Total}   & \$15           & \$15            \\
    \end{tabular}

    \vspace{1em}

    We make the following closing entry:

    \vspace{1em}

    \begin{tabular}{llll}
        \textbf{Date} & \textbf{Account}           & \textbf{Dr} & \textbf{Cr}  \\
        \hline
                      & \textit{in thousands}      &             &              \\
        Dec 31, 2023  & Revenue                    & \$15        &              \\
                      & \textbf{Retained Earnings} &             & \textbf{\$5} \\
                      & $\quad$Expense             &             & \$5          \\
                      & $\quad$Dividends           &             & \$5          \\
    \end{tabular}
\end{tcolorbox}