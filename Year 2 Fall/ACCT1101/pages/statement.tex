\section{Financial Statements}
\label{sec:financial_statements}

The financial statements are the primary means of communication between a company and its stakeholders. It allows them to make informed decisions about providing resources to the entity (e.g. investing, lending).

\begin{knBox}
    {Pervasive Cost-Benefit Constraint}
    The benefits of providing finanical information should outweigh its costs.
\end{knBox}

\begin{knBox}
    {Formatting}
    The following are the \underline{required} headers and footers for financial statements:
    \begin{itemize}
        \item Name of the company
        \item Title (type of statement)
        \item Specific date covered (At December 31, 2025)
        \item Unit of measure (In millions of dollars)
        \item Bottom: \textit{The notes are an integral part of the financial statements.}
    \end{itemize}
    Some conventions for styling:
    \begin{itemize}
        \item Dollar sign for first and total amounts
        \item Line above sub-totals
        \item Double underline for totals
        \item Brackets for negative numbers
    \end{itemize}
\end{knBox}

\begin{theorem}
    {Classified vs. Unclassified Financial Statements}
    A classified financial statement would break accounts into different classes (e.g. current / non-current A\&L), whereas an unclassified financial statement would list all accounts in the same class together.
\end{theorem}

We can build financial statements by referencing your adjusted trial balance.

\subsection{Relations between account types}
\begin{theorem}
    {Extended accounting equation}
    The following equations are based on the definition of the individual items:
    \begin{center}
        \textbf{A}ssets = \textbf{L}iabilities + \textbf{E}quity

        \textit{Retained Earnings} = \textbf{R}evenue - \textbf{E}xpenses

        $\Delta$\textbf{E}quity = \textit{Retained Earnings} - \textbf{D}ividends
    \end{center}
    \textit{Retained Earnings} here is the \textbf{Net income} generated by the business, and what's left of it.

    These relationships \& equations are demonstrated in their respective financial statements:
    \begin{enumerate}
        \item \nameref{sec:balance_sheet}
        \item \nameref{sec:income_statement}
        \item \nameref{sec:statement_se}
    \end{enumerate}
    \label{thm:relations}
\end{theorem}

\subsection{Income Statement}
\label{sec:income_statement}

Principle: \hyperref[thm:relations]{\textbf{RE} = \textbf{R} - \textbf{E}}

Reports profit / loss \underline{over a period of time}. \textit{Income before / after taxes must be displayed separately}

\small
\begin{tcolorbox}[colframe=black,colback=white,title=Example Income Statement (Classified by Operating \& Non-operating)]
    \begin{center}
        \textbf{XYZ Corporation}\\
        \textbf{Income Statement}\\
        For the Year Ended December 31, 2025\\
        (In millions of dollars)
    \end{center}

    Sales Revenue \hfill \underline{\$5}
    \begin{itemize}[label={}, leftmargin=*]
        \item Less: Credit card discounts \hfill (1)
        \item Less: Cost of goods sold \hfill (1)
    \end{itemize}
    Gross profit \hfill 3

    \textbf{Operating Expenses}
    \begin{itemize}[label={}, leftmargin=*]
        \item Supplies Expense \hfill \$1
        \item Loss on sale of assets \hfill \underline{1}
    \end{itemize}
    Total operating expenses \hfill 2

    \textbf{Income from operations} \hfill \underline{\underline{\$1}}


    Other items:
    \begin{itemize}[label={}, leftmargin=*]
        \item Interest Revenue \hfill \$3
        \item Dividend Expense \hfill \underline{1}
    \end{itemize}
    \textbf{Income before taxes} \hfill \underline{\underline{\$3}}

    Tax Expense \hfill \underline{1}


    \textbf{Net Income} \hfill \underline{\underline{\$2}}

    Earnings per share (2 million shares) \hfill \underline{\underline{\$1}}

    \vspace{1em}

    \textit{\footnotesize{The notes are an integral part of these financial statements. (Operating revenue can simplify be "Sales Revenue" as there is only one item, but expanded for clarity.)}}
\end{tcolorbox}

\begin{knBox}
    {Key items of the Income Statement}
    \begin{itemize}
        \item Operating R\&E $\rightarrow$ Income from operations
        \item Other Revenue / Expense $\rightarrow$ Income before taxes
        \item Tax Expense $\rightarrow$ Net Income \& EPS
    \end{itemize}
\end{knBox}

\begin{knBox}
    {Gross profit}
    The difference between \textbf{Sales Revenue} and \textbf{Cost of Goods Sold}.
\end{knBox}

\subsection{Statement of Stockholder's Equity}
\label{sec:statement_se}

Principles:

\begin{itemize}
    \item \hyperref[def:sale_of_stock]{Beginning \textbf{CS} + \textbf{Stock Issurance} = Ending \textbf{CS}}
    \item \hyperref[thm:relations]{Beginnging \textbf{RE} + \textbf{NI} - \textbf{D} = Ending \textbf{RE}}
\end{itemize}


Reports changes in stockholder's equity over a period of time.

\small

\begin{tcolorbox}[colframe=black,colback=white,title=Example Statement of Stockholder's Equity]
    \begin{center}
        \textbf{XYZ Corporation}\\
        \textbf{Statement of Stockholder's Equity}\\
        For the Year Ended December 31, 2025\\
        (In millions of dollars)
    \end{center}

    \begin{tabular}{lcccc}
                                            & \textbf{CS}      & \textbf{RE}       & \textbf{APC}    & \textbf{Total}                \\
        \textbf{Balance, January 1, 2025}   & \$50             & \$100             & \$2             & \$152                         \\
        Net Income                          & --               & 25                & --              & 25                            \\
        Dividends                           & --               & (10)              & --              & (10)                          \\
        Stock Issuance                      & 20               & --                & 2               & 22                            \\
        \textbf{Balance, December 31, 2025} & \underline{\$70} & \underline{\$115} & \underline{\$4} & \underline{\underline{\$189}} \\
    \end{tabular}

    \vspace{1em}

    \textit{\footnotesize{The notes are an integral part of these financial statements.}}
\end{tcolorbox}

\normalsize

\subsection{Balance Sheet}
\label{sec:balance_sheet}

Principle: \hyperref[thm:relations]{\textbf{A} = \textbf{L} + \textbf{E}}

The balance sheet is prepared \textbf{after creating the \nameref{sec:closing_entries}}.

Reports the financial position of a company at a \underline{specific point in time}. Helpful for demonstrating the \hyperref[def:current_ratio]{current ratio}.

\small
\begin{tcolorbox}[colframe=black,colback=white,title=Example Balance Sheet (Classified by Current \& Non-current)]
    \begin{center}
        \textbf{XYZ Corporation}\\
        \textbf{Balance Sheet}\\
        At December 31, 2025\\
        (In millions of dollars)
    \end{center}

    \begin{multicols}{2}
        \textbf{Assets}\\
        Current Assets:
        \begin{itemize}
            \item Cash and Cash Equivalents \hfill \$14
            \item Accounts Receivable \hfill 3
            \item Inventory \hfill \underline{7}
        \end{itemize}
        Total Current Assets \hfill 24

        Non-Current Assets:
        \begin{itemize}
            \item Property, Plant, and Equipment \hfill 35
            \item Intangible Assets \hfill \underline{5}
        \end{itemize}
        Total Non-Current Assets \hfill 40
        \vfill
        \textbf{Total Assets} \hfill \underline{\underline{\$64}}


        \columnbreak

        \textbf{Liabilities and Stockholders' Equity}\\
        Current Liabilities:
        \begin{itemize}
            \item Accounts Payable \hfill \$4
            \item Short-term Debt \hfill 2
        \end{itemize}
        Total Current Liabilities \hfill \underline{6}

        Non-Current Liabilities:
        \begin{itemize}
            \item Long-term Debt \hfill 20
            \item Deferred Tax Liabilities \hfill \underline{5}
        \end{itemize}
        Total Non-Current Liabilities \hfill 25

        Total Liabilities \hfill \underline{31}

        Stockholders' Equity:
        \begin{itemize}
            \item Common Stock \hfill 20
            \item Less: Treasury stock \hfill (4)
            \item Retained Earnings \hfill \underline{17}
        \end{itemize}
        Total Stockholders' Equity \hfill 33\\
        {\textbf{Total Liabilities and Stockholders' Equity}} \hfill \underline{\underline{\$64}}
    \end{multicols}

    \vspace{1em}

    \textit{\footnotesize{The notes are an integral part of these financial statements.}}
\end{tcolorbox}

\normalsize

It is called a \textbf{balance sheet} because the \textbf{left side} (assets) is equal to the \textbf{right side} (liabilities + equity).

\subsection{Cash flow statement}

\subsubsection{Cash flows}

\begin{theorem}
    {Identifying types of cash flows activities}
    The following is a general way to categorize cash flows:
    \begin{itemize}
        \item \textbf{Financing}: Changes in shares, borrowing money or repayment of long term debt
        \item \textbf{Investing}: The purchase / sale of long-term assets that sit outside of the businessses' core operations
        \item \textbf{Operating}: Revenue generating activities of a business
    \end{itemize}
    An entry has \textbf{no effect} on cash flow if it doesn't involve cash.

    The direction of cash flow: cash is \textbf{debited} (cash outflow) or \textbf{credited} (cash inflow).
\end{theorem}

\subsubsection{Information needed for preparing cash flow statement}
\begin{itemize}
    \item \textit{Comparative} balance sheets (showing current and previous years)
    \item Income statement
\end{itemize}

\subsubsection{Preparing statement with indirect method}

Principle: Calculate cash flows and check with the change in cash balance.

We generally use \textbf{indirect} method, which is more convenient to prepare. The only difference between the indirect and direct method is the operating activities section. We're just going to focus on the indirect method.

We can calculate the \textbf{cash flow from operating activities by}:
\[ \mathcolorbox{lime}{\text{Net Income}} + \mathcolorbox{pink}{\text{Non-cash expenses} + \text{Gain/Loss on sale of non-current A}} - \mathcolorbox{yellow}{\Delta\text{cA\&cL}} \]
Which we will do inside the statement. The general steps to prepare the statement are:
\begin{itemize}
    \item Calculate change and label type of cash flow (FIO) on the balance sheet, as well as for any additional information
    \item Identify the non-cash expenses (depreciation, amortization)
    \item Identify if there are any gains or losses on the sale of non-current assets
    \item Use the changes in [O] and \textbf{reverse their effects} on the cash flow statement\\
          (e.g. $\uparrow A \rightarrow -ve,\quad\downarrow L \rightarrow -ve$ )
    \item Fill in the rest
\end{itemize}

Note that we use \textbf{payments} and \textbf{proceeds} for financing activities to describe cash outflows and inflows, respectively.

\small

\begin{tcolorbox}[colframe=black,colback=white,title=Example Statement of Cash Flow]
    \begin{center}
        \textbf{XYZ Corporation}\\
        \textbf{Statement of Cash Flows}\\
        For the Year Ended December 31, 2025\\
        (In millions of dollars)
    \end{center}

    \textbf{Cash flows from operating activities}
    \begin{itemize}[label={}, leftmargin=*]
        \item \colorbox{lime}{Net Income} \hfill \$2
        \item \colorbox{pink}{Adjustments for non-cash items}:
              \begin{itemize}[label={}, leftmargin=*]
                  \item Depreciation Expense \hfill 1
                  \item Amortization Expense \hfill 1
                  \item Loss on Sale of Equipment \hfill 1
              \end{itemize}
        \item \colorbox{yellow}{Changes in current asset and liabilities}:
              \begin{itemize}[label={}, leftmargin=*]
                  \item Increase in Accounts Receivable \hfill (1)
                  \item Decrease in Inventory \hfill 2
                  \item Increase in Accounts Payable \hfill \underline{1}
              \end{itemize}
    \end{itemize}
    Net cash used by operating activities \hfill 7

    \textbf{Cash flows from investing activities}
    \begin{itemize}[label={}, leftmargin=*]
        \item Acquisition of businesses \hfill (5)
        \item Purchase of Equipment \hfill (5)
        \item Sale of Equipment \hfill \underline{2}
    \end{itemize}
    Net cash used in investing activities \hfill (8)

    \textbf{Cash flows from financing activities}
    \begin{itemize}[label={}, leftmargin=*]
        \item Issuance of Common Stock \hfill 4
        \item Payment of Dividends \hfill (1)
        \item Payments on revolving line of bank credit \hfill (1)
        \item Purchases of treasury stock \hfill (1)
        \item Proceeds from stock issued under share-based compensation plans \hfill \underline{1}
    \end{itemize}
    Net cash used by financing activities \hfill 2

    \textbf{Net increase (decrease) in cash} \hfill \underline{\underline{\$1}}

    Cash at beginning of period \hfill 13

    Cash at end of period \hfill 14

    \vspace{1em}

    \textit{\footnotesize{The notes are an integral part of these financial statements.}}
\end{tcolorbox}

\subsection{Order of creation}

Note the importance of the order of creation of the statements as listed:
\begin{enumerate}
    \item Income Statement: Provides the \textbf{Net Income} $\downarrow$
    \item Statement of Stockholder's Equity: Provides the \textbf{Ending RE} $\downarrow$
    \item Balance Sheet: Provides the \textbf{Ending A, L, E} to compare with the previous year $\downarrow$
    \item Statement of Cash Flows:
\end{enumerate}