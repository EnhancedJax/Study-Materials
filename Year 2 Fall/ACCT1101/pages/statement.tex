\section{Financial Statements}
\label{sec:financial_statements}

The financial statements are the primary means of communication between a company and its stakeholders. It allows them to make informed decisions about providing resources to the entity (e.g. investing, lending).

\begin{knBox}
    {Pervasive Cost-Benefit Constraint}
    The benefits of providing finanical information should outweigh its costs.
\end{knBox}

\begin{knBox}
    {Formatting}
    The following are the \underline{required} headers and footers for financial statements:
    \begin{itemize}
        \item Name of the company
        \item Title (type of statement)
        \item Specific date covered (At December 31, 2025)
        \item Unit of measure (In millions of dollars)
        \item Bottom: \textit{The notes are an integral part of the financial statements.}
    \end{itemize}
    Some conventions for styling:
    \begin{itemize}
        \item Dollar sign for first and total amounts
        \item Line above sub-totals
        \item Double underline for totals
        \item Brackets for negative numbers
    \end{itemize}
\end{knBox}

\begin{theorem}
    {Classified vs. Unclassified Financial Statements}
    A classified financial statement would break accounts into different classes (e.g. current / non-current A\&L), whereas an unclassified financial statement would list all accounts in the same class together.
\end{theorem}

We can build financial statements by referencing your adjusted trial balance.

\subsection{Relations between account types}
\begin{theorem}
    {Extended accounting equation}
    The following equations are based on the definition of the individual items:
    \begin{center}
        \textbf{A}ssets = \textbf{L}iabilities + \textbf{E}quity

        \textit{Retained Earnings} = \textbf{R}evenue - \textbf{E}xpenses

        $\Delta$\textbf{E}quity = \textit{Retained Earnings} - \textbf{D}ividends
    \end{center}
    \textit{Retained Earnings} here is the \textbf{Net income} generated by the business, and what's left of it.

    These relationships \& equations are demonstrated in their respective financial statements:
    \begin{enumerate}
        \item \nameref{sec:balance_sheet}
        \item \nameref{sec:income_statement}
        \item \nameref{sec:statement_se}
    \end{enumerate}
    \label{thm:relations}
\end{theorem}

\subsection{Income Statement}
\label{sec:income_statement}

Principle: \hyperref[thm:relations]{\textbf{RE} = \textbf{R} - \textbf{E}}

Reports profit / loss \underline{over a period of time}. \textit{Income before / after taxes must be displayed separately}

\small
\begin{tcolorbox}[colframe=black,colback=white,title=Example Income Statement (Classified by Operating \& Non-operating)]
    \begin{center}
        \textbf{XYZ Corporation}\\
        \textbf{Income Statement}\\
        For the Year Ended December 31, 2025\\
        (In millions of dollars)
    \end{center}

    \textbf{Operating Revenue}
    \begin{itemize}[label={}, leftmargin=*]
        \item Sales Revenue \hfill \underline{\$3}
    \end{itemize}

    Total operating revenue \hfill 3

    \textbf{Operating Expenses}
    \begin{itemize}[label={}, leftmargin=*]
        \item Supplies Expense \hfill \$1
        \item Training Expenses \hfill \underline{1}
    \end{itemize}
    Total operating expenses \hfill 2

    \textbf{Income from operations} \hfill \underline{\underline{\$1}}


    Other items:
    \begin{itemize}[label={}, leftmargin=*]
        \item Interest Revenue \hfill \$3
        \item Dividend Expense \hfill \underline{1}
    \end{itemize}
    \textbf{Income before taxes} \hfill \underline{\underline{\$3}}

    Tax Expense \hfill \underline{1}


    \textbf{Net Income} \hfill \underline{\underline{\$2}}

    Earnings per share (2 million shares) \hfill \underline{\underline{\$1}}

    \vspace{1em}

    \textit{\footnotesize{The notes are an integral part of these financial statements. (Operating revenue can simplify be "Sales Revenue" as there is only one item, but expanded for clarity.)}}
\end{tcolorbox}

\begin{knBox}
    {Key items of the Income Statement}
    \begin{itemize}
        \item Operating R\&E $\rightarrow$ Income from operations
        \item Other Revenue / Expense $\rightarrow$ Income before taxes
        \item Tax Expense $\rightarrow$ Net Income \& EPS
    \end{itemize}
\end{knBox}

\subsection{Balance Sheet}
\label{sec:balance_sheet}

Principle: \hyperref[thm:relations]{\textbf{A} = \textbf{L} + \textbf{E}}

The balance sheet is prepared \textbf{after creating the \nameref{sec:closing_entries}}.

Reports the financial position of a company at a \underline{specific point in time}. Helpful for demonstrating the \hyperref[def:current_ratio]{current ratio}.

\small
\begin{tcolorbox}[colframe=black,colback=white,title=Example Balance Sheet (Classified by Current \& Non-current)]
    \begin{center}
        \textbf{XYZ Corporation}\\
        \textbf{Balance Sheet}\\
        At December 31, 2025\\
        (In millions of dollars)
    \end{center}

    \begin{multicols}{2}
        \textbf{Assets}\\
        Current Assets:
        \begin{itemize}
            \item Cash and Cash Equivalents \hfill \$14
            \item Accounts Receivable \hfill 3
            \item Inventory \hfill \underline{7}
        \end{itemize}
        Total Current Assets \hfill 24

        Non-Current Assets:
        \begin{itemize}
            \item Property, Plant, and Equipment \hfill 35
            \item Intangible Assets \hfill \underline{5}
        \end{itemize}
        Total Non-Current Assets \hfill 40
        \vfill
        \textbf{Total Assets} \hfill \underline{\underline{\$64}}


        \columnbreak

        \textbf{Liabilities and Stockholders' Equity}\\
        Current Liabilities:
        \begin{itemize}
            \item Accounts Payable \hfill \$4
            \item Short-term Debt \hfill 2
        \end{itemize}
        Total Current Liabilities \hfill \underline{6}

        Non-Current Liabilities:
        \begin{itemize}
            \item Long-term Debt \hfill 20
            \item Deferred Tax Liabilities \hfill \underline{5}
        \end{itemize}
        Total Non-Current Liabilities \hfill 25

        Total Liabilities \hfill \underline{31}

        Stockholders' Equity:
        \begin{itemize}
            \item Common Stock \hfill 20
            \item Retained Earnings \hfill \underline{13}
        \end{itemize}
        Total Stockholders' Equity \hfill 33\\
        {\textbf{Total Liabilities and Stockholders' Equity}} \hfill \underline{\underline{\$64}}
    \end{multicols}

    \vspace{1em}

    \textit{\footnotesize{The notes are an integral part of these financial statements.}}
\end{tcolorbox}

\normalsize

It is called a \textbf{balance sheet} because the \textbf{left side} (assets) is equal to the \textbf{right side} (liabilities + equity).

\newpage

\subsection{Statement of Stockholder's Equity}
\label{sec:statement_se}

Principles:

\begin{itemize}
    \item \hyperref[def:cs]{Beginning \textbf{CS} + \textbf{Stock Issurance} = Ending \textbf{CS}}
    \item \hyperref[thm:relations]{Beginnging \textbf{RE} + \textbf{NI} - \textbf{D} = Ending \textbf{RE}}
\end{itemize}


Reports changes in stockholder's equity over a period of time.

\small

\begin{tcolorbox}[colframe=black,colback=white,title=Example Statement of Stockholder's Equity]
    \begin{center}
        \textbf{XYZ Corporation}\\
        \textbf{Statement of Stockholder's Equity}\\
        For the Year Ended December 31, 2025\\
        (In millions of dollars)
    \end{center}

    \begin{tabular}{lcccc}
                                            & \textbf{CS}      & \textbf{RE}       & \textbf{APC}    & \textbf{Total}                \\
        \textbf{Balance, January 1, 2025}   & \$50             & \$100             & \$2             & \$152                         \\
        Net Income                          & --               & 25                & --              & 25                            \\
        Dividends                           & --               & (10)              & --              & (10)                          \\
        Stock Issuance                      & 20               & --                & 2               & 22                            \\
        \textbf{Balance, December 31, 2025} & \underline{\$70} & \underline{\$115} & \underline{\$4} & \underline{\underline{\$189}} \\
    \end{tabular}

    \vspace{1em}

    \textit{\footnotesize{The notes are an integral part of these financial statements.}}
\end{tcolorbox}

\normalsize

\subsection{Cash flow statement}

\subsubsection{Cash flows}

\begin{theorem}
    {Identifying types of cash flows activities}
    The following is a general way to categorize cash flows:
    \begin{itemize}
        \item \textbf{Financing}: Equality-accounts related / Cash borrowed / Repay dividends
        \item \textbf{Investing}: Non-current assets (long-term)
        \item \textbf{Operating}: Day to day activities
    \end{itemize}
    An entry has \textbf{no effect} on cash flow if it doesn't involve cash.

    The direction of cash flow: cash is \textbf{debited} (cash outflow) or \textbf{credited} (cash inflow).
\end{theorem}

\newpage
