\section{Financial Accounting}

The process of identifying, recording, summarizing \& analyzing an entity's financial transactions and reporting them in financial statements to its existing \& potential investors, lenders and creditors.

\subsection{The accounting cycle}

\begin{definition}
    {Accounting period}
    The period of time over which financial statements are prepared.
\end{definition}

\begin{knBox}
    {Entries}
    A record of a transaction. In the accounting cycle, there are 3 types of entries:
    \begin{itemize}
        \item \textbf{Journal Entry}
        \item \textbf{Adjusting Entry}
        \item \textbf{Closing Entry}
    \end{itemize}
\end{knBox}

\begin{center}
    \begin{tikzpicture}
        \node[anchor=center] (list) {
            \begin{minipage}{0.4\textwidth}
                \begin{itemize}
                    \item \textit{Record (During the period)} \ref{sec:record}
                          \begin{enumerate}
                              \item Identify transactions \ref{subsec:identify}
                              \item Prepare journal entries \ref{subsec:prepare}
                              \item Post to general ledger \ref{subsec:post}
                          \end{enumerate}
                    \item \textit{Trial (At the end of the period)} \ref{sec:trial}
                          \begin{enumerate}[resume]
                              \item Unadjusted trial balance
                              \item Post adjusting entries \ref{subsec:post_adjust}
                              \item Adjusted trial balance
                          \end{enumerate}
                    \item \textit{Post}
                          \begin{enumerate}[resume]
                              \item Create finanical statements \ref{sec:financial_statements} \& Post closing entries \ref{sec:closing_entries}
                          \end{enumerate}
                \end{itemize}
            \end{minipage}
        };

        \draw[->, thick] (list.north west) -- (list.south west);

        \draw[->, thick] (list.south east) to[bend right] (list.north east);
    \end{tikzpicture}
\end{center}

During the period, the cycle records all your transactions accordingly by the \hyperref[thm:cash_basis]{Cash Basis of Accounting}. At the end of a period, we test for errors and respect our transactions by the \hyperref[thm:accrual_basis]{Accural Method of Accounting}. Finally, we create finanical statements that will be published, so that stakeholders can make decisions based on the financial information.

\subsection{Accounting principles}
Accounting principles are to make sure that companies' statements are \textit{true and fair}.
\begin{theorem}
    {Accural method of accounting}
    Most accounting rules, like the \textbf{G}enerally \textbf{A}ccepted \textbf{A}ccounting \textbf{P}rinciples (GAAP), are based on the \textbf{accrual basis of accounting}.
    \vspace{0.5em}

    This mean that \underline{revenue is recognized when it is \textbf{earned}},\\and \underline{expenses are recognized when they are \textbf{incurred}}, regardless of when cash is received or paid.
    \label{thm:accrual_basis}
\end{theorem}
\begin{knBox}
    {Cash basis of accounting}
    This is a method of accounting where \underline{revenue is recognized when cash is received} and \underline{expenses are recognized when cash is paid}.
    \label{thm:cash_basis}
\end{knBox}

The following gives a list of other principles:
\begin{table}[H]
    \centering
    \begin{tabular}{|p{0.3\textwidth}|p{0.6\textwidth}|}
        \hline
        \textbf{Principle}  & \textbf{Description}                                                  \\
        \hline
        Comparability       & Financial statements must be comparable period to period              \\
        \hline
        Conservatism        & Considers all risks | strict rules                                    \\
        \hline
        Consistency         & Same accounting methods year to year                                  \\
        \hline
        Constraints         & Information has a cost/benefit and is material                        \\
        \hline
        Cost principle      & Keep costs at purchase price or lower (lower of cost or market)       \\
        \hline
        Economic entity     & Maintain separate records for each entity                             \\
        \hline
        Full disclosure     & Provides detailed information in addition to financial statements     \\
        \hline
        Going concern       & Assume business is going to and has capability to continue            \\
        \hline
        Matching            & Recognize cost the same time as benefit                               \\
        \hline
        Materiality         & Significance to the overall financial picture                         \\
        \hline
        Monetary unit       & Currency is used to record transactions and is assumed to be constant \\
        \hline
        Relevance           & Financial reporting has predictive, feedback, and timeliness value    \\
        \hline
        Reliability         & Financial reporting is neutral, valid, and verifiable                 \\
        \hline
        Revenue recognition & Conditions of how an organization records revenue                     \\
        \hline
        Time period         & Report financial activity in specific time periods                    \\
        \hline
    \end{tabular}
    \caption{Accounting Principles}
\end{table}


\newpage
