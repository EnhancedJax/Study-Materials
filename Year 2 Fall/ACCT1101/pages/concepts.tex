\section{Concepts Dump}

\begin{theorem}
    {Internal control of Cash}
    Cash is the most vulnerable to \textbf{theft and fraud}. Therefore companies usually employ the following procedures to handle cash:
    \begin{itemize}
        \item \textbf{Segregation of duties}: Different people are responsible for handling cash, recording transactions, and reconciling the bank statements.
        \item \textbf{Authorization}: All cash transactions must be approved by a supervisor.
        \item \textbf{Reconciliation}: Bank accounts and cash accounts to be checked regularly.
        \item \textbf{Physical control}: Secure cash under strict control.
        \item \textbf{Documentation}: All cash transactions should be properly documented.
    \end{itemize}
\end{theorem}

\begin{theorem}
    {Accounting policy: Free on board (FOB)}
    The point where the \textbf{ownership of the goods is transferred} from the seller to the buyer.
    \begin{itemize}
        \item FOB Destination - When arrived at destination
        \item FOB Shipping point - When left the origin
    \end{itemize}
    \textit{We usually declare our policy at the notes to the financial statements.}
\end{theorem}

\subsection{Reporting long-term liabilities}

\begin{theorem}
    {Report contingent liabilities}
    A contingent liability is a \textbf{potential liability} that arises from past events and about which a company cannot report with confidence.

    \begin{tabular}{|p{0.3\textwidth}|p{0.5\textwidth}|}
        \hline
        \textbf{Likelihood}      & \textbf{Treatment}     \\
        \hline
        Probable and estimable   & Record as liability    \\
        \hline
        Probable but inestimable & Disclose in notes only \\
        \hline
        Reasonably possible      & Disclose in notes only \\
        \hline
        Remote                   & No disclosure needed   \\
        \hline
    \end{tabular}
\end{theorem}

\begin{theorem}
    {Present value of lump sum}
    A lump sum is a single payment made at a specific point in time.
    \[PV = FV \times (1+i)^{-n}\]
    Where $i$ is the interest rate and $n$ is the number of periods. $FV$ is the future value of the lump sum.
\end{theorem}

\begin{theorem}
    {Present values of Annuities}
    An annuity is a series of equal payments made at regular intervals.
    \[PV = PMT \times T(n, i)\]
    Where $PMT$ is the payment per period,
    $n$ is the number of periods, and
    $i$ is the interest rate.
    The ratio is the present value factor from the \href{https://www.double-entry-bookkeeping.com/wp-content/uploads/present-value-annuity-tables-v-1.0.jpg}{present value annuity tables}.
\end{theorem}
