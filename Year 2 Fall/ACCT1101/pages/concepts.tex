\section{Concepts Dump}

\begin{theorem}
    {Internal control of Cash}
    Cash is the most vulnerable to \textbf{theft and fraud}. Therefore companies usually employ the following procedures to handle cash:
    \begin{itemize}
        \item \textbf{Segregation of duties}: Different people are responsible for handling cash, recording transactions, and reconciling the bank statements.
        \item \textbf{Authorization}: All cash transactions must be approved by a supervisor.
        \item \textbf{Reconciliation}: Bank accounts and cash accounts to be checked regularly.
        \item \textbf{Physical control}: Secure cash under strict control.
        \item \textbf{Documentation}: All cash transactions should be properly documented.
    \end{itemize}
\end{theorem}

\begin{theorem}
    {Accounting policy: Free on board (FOB)}
    The point where the \textbf{ownership of the goods is transferred} from the seller to the buyer.
    \begin{itemize}
        \item FOB Destination - When arrived at destination
        \item FOB Shipping point - When left the origin
    \end{itemize}
    \textit{We usually declare our policy at the notes to the financial statements.}
\end{theorem}

\subsection{Reporting long-term liabilities}

\begin{theorem}
    {Report contingent liabilities}
    A contingent liability is a \textbf{potential liability} that arises from past events and about which a company cannot report with confidence.

    \begin{tabular}{|p{0.3\textwidth}|p{0.5\textwidth}|}
        \hline
        \textbf{Likelihood}             & \textbf{Treatment}     \\
        \hline
        Probable \textbf{and estimable} & Record as liability    \\
        \hline
        Reasonably possible             & Disclose in notes only \\
        \hline
        Remote                          & No disclosure needed   \\
        \hline
    \end{tabular}
\end{theorem}

\begin{theorem}
    {Present value of lump sum}
    A lump sum is a single payment made at a specific point in time.
    \[PV = FV \times (1+i)^{-n}\]
    Where $i$ is the interest rate and $n$ is the number of periods. $FV$ is the future value of the lump sum.
\end{theorem}

\begin{theorem}
    {Present values of Annuities}
    An annuity is a series of equal payments made at regular intervals.
    \[PV = PMT \times T(n, i)\]
    Where $PMT$ is the payment per period,
    $n$ is the number of periods, and
    $i$ is the interest rate.
    The ratio is the present value factor from the \href{https://www.double-entry-bookkeeping.com/wp-content/uploads/present-value-annuity-tables-v-1.0.jpg}{present value annuity tables}.
\end{theorem}

\subsection{Mid-term / exam formats}

Mid-terms:
\begin{itemize}
    \item MC questions
    \item Short problem-solving questions that focuses on a single concept usually
    \item Prepare a statement (balance sheet / income statement)
    \item Calculate ratios either one: (liquidity + NPM + Fixed-asset turnover)
\end{itemize}

Finals:
\begin{itemize}
    \item 1- 5 MC - 20 marks in total
    \item 5 Problem solving - Shorter than assignments, longer than mid-terms (subparts)
    \item 1 Short essay question\\
          Concept centric questions, on accounting concept that covered in class, such as:
          \begin{itemize}
              \item why are there multiple inventory costing methods
              \item what are the results from different inventory costing methods
              \item financial statements - what are they trying to show
              \item different kinds of depreciation methods - when straight line, when double decline method etc.
          \end{itemize}
          Around 100 - 300 words. They wouldn't be related to the ratios / financial statement analysis.
\end{itemize}
