\documentclass{article}

% ------------------------------------ %
%             Document Info            %
% ------------------------------------ %

\usepackage{../../../LaTeX-Preamables/Assign}

\begin{document}
\newcommand{\documentcourse}{IIMT2641}
\newcommand{\documentnumber}{4}

% ------------------------------------ %
%                Header                %
% ------------------------------------ %

\begin{minipage}{0.07\textwidth}
    \includegraphics[width=\linewidth]{../../../LaTeX-Preamables/LaTeX-Templates/HKULOGO256.png}
\end{minipage}
\hspace{0.02\textwidth}
\begin{minipage}{0.55\textwidth}
    \documentcourse

    Assignment \documentnumber

    SID: 3036268218
\end{minipage}
\begin{minipage}{0.35\textwidth}
    \begin{flushright}
        Jax

        \jobname.pdf

        \today
    \end{flushright}
\end{minipage}

\vspace{0.5cm}

\hrule

% ------------------------------------ %
%                Content               %
% ------------------------------------ %

\begin{enumerate}[label=\alph*.]
      \item
            \begin{enumerate}[label=\roman*.]
                  \item The baseline accuracy (predicting all loans paid) is $0.8399$.

                  \item
                        \begin{verbatim}
Call:
glm(formula = NotFullyPaid ~ ., family = "binomial", data = Train)

Coefficients:
                  Estimate Std. Error z value Pr(>|z|)    
(Intercept)     8.172e+00  1.559e+00   5.240 1.60e-07 ***
CreditPolicy   -1.343e-01  1.027e-01  -1.307 0.191145    
Purpose.CC     -5.267e-01  1.243e-01  -4.238 2.25e-05 ***
Purpose.DC     -3.785e-01  8.734e-02  -4.334 1.47e-05 ***
Purpose.Edu    -5.956e-02  1.873e-01  -0.318 0.750470    
Purpose.MP     -3.892e-01  1.905e-01  -2.044 0.040993 *  
Purpose.SB      5.331e-01  1.338e-01   3.984 6.79e-05 ***
IntRate         1.769e+00  2.058e+00   0.859 0.390160    
Installment     1.215e-03  2.091e-04   5.809 6.28e-09 ***
LogAnnualInc   -4.150e-01  7.215e-02  -5.751 8.87e-09 ***
Dti             8.940e-04  5.491e-03   0.163 0.870674    
Fico           -8.703e-03  1.703e-03  -5.109 3.24e-07 ***
DaysWithCrLine  7.414e-06  1.593e-05   0.465 0.641586    
RevolBal        4.307e-06  1.181e-06   3.647 0.000265 ***
RevolUtil       3.157e-03  1.547e-03   2.041 0.041269 *  
InqLast6mths    1.126e-01  1.668e-02   6.750 1.48e-11 ***
Delinq2yrs     -5.217e-02  6.372e-02  -0.819 0.412896    
PubRec          3.569e-01  1.133e-01   3.150 0.001631 ** 
---
Signif. codes:  0 ‘***’ 0.001 ‘**’ 0.01 ‘*’ 0.05 ‘.’ 0.1 ‘ ’ 1

(Dispersion parameter for binomial family taken to be 1)

      Null deviance: 5896.6  on 6704  degrees of freedom
Residual deviance: 5480.4  on 6687  degrees of freedom
AIC: 5516.4

Number of Fisher Scoring iterations: 5
                        \end{verbatim}

                        \textbf{Significant Predictors}: \texttt{Purpose.CC}, \texttt{Purpose.DC}, \texttt{Purpose.MP}, \texttt{Purpose.SB}, \texttt{Installment}, \texttt{LogAnnualInc}, \texttt{Fico}, \texttt{RevolBal}, \texttt{RevolUtil}, \texttt{InqLast6mths}, \texttt{and PubRec}

                        The model shows that various \textit{loan-specific details} and \textit{borrower characteristics} have significant impacts on the probability of a loan not being fully paid.


                  \item The difference in logit values between a loan with \texttt{FICO} 700 and 710 is \texttt{0.0870}.

                  \item The table prints the confusion matrix:
                        \begin{table}[H]
                              \centering
                              \begin{tabular}{ccc}
                                      & FALSE & TRUE \\
                                    \hline
                                    0 & 2396  & 17   \\
                                    1 & 451   & 9
                              \end{tabular}
                        \end{table}
                        The model accuracy on the test set with threshold 0.5 is $\frac{2396 + 9}{2396 + 17 + 451 + 9}= 0.8371 < 0.8399$, which is slightly worse than the baseline model.

            \end{enumerate}

      \item
            \begin{enumerate}[label=\roman*.]
                  \item The simple model using only interest rate shows IntRate is highly significant (p $<$ 0.001), however it is not significant in the original model. This is probbly because the original model has more variables, where they might have a more signficiant effect on the outcome.

                  \item The highest predicted probability of non-payment is $0.4440$, and no loans would be predicted as unpaid using a 0.5 threshold.
            \end{enumerate}
\end{enumerate}

\end{document}