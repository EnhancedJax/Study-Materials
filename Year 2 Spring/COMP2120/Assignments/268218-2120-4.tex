\documentclass{article}

% ------------------------------------ %
%             Document Info            %
% ------------------------------------ %

\usepackage{../../../LaTeX-Preamables/Assign}

\begin{document}
\newcommand{\documentcourse}{COMP2120}
\newcommand{\documentnumber}{4}

% ------------------------------------ %
%                Header                %
% ------------------------------------ %

\begin{minipage}{0.07\textwidth}
    \includegraphics[width=\linewidth]{../../../LaTeX-Preamables/LaTeX-Templates/HKULOGO256.png}
\end{minipage}
\hspace{0.02\textwidth}
\begin{minipage}{0.55\textwidth}
    \documentcourse

    Assignment \documentnumber

    SID: 3036268218
\end{minipage}
\begin{minipage}{0.35\textwidth}
    \begin{flushright}
        Jax

        \jobname.pdf

        \today
    \end{flushright}
\end{minipage}

\vspace{0.5cm}

\hrule

% ------------------------------------ %
%                Content               %
% ------------------------------------ %
\begin{enumerate}
    \item The purpose of \texttt{SQ} is to calculate the \textbf{square} of `R10`, input is R10, output is R11. The proc uses R12 and R13, need to save them on entry and restore them when exit.
    \item Using \texttt{[]} instead of \texttt{mem[]} for simplicity. For \texttt{CALL}:
          \begin{verbatim}
Fetch:
---
MAR <- PC
IR <- [MAR]
Increment PC by 4
    
Execute:  
---
MAR <- PC
MBR <- [MAR]
MAR <- MBR​
Increment PC by 4
TEMP <- MAR
Decrement SP by 4
MAR <- SP
MBR <- PC
[MAR] <- MBR
MAR <- TEMP
PC <- MAR 
\end{verbatim}
          For \texttt{RET}:
          \begin{verbatim}
Fetch:
---
MAR <- PC
IR <- [MAR]
Increment PC by 4

Execute:  
---
MAR <- SP
Increment SP by 4
MBR <- [MAR]
PC <- MBR
\end{verbatim}
    \item The following is the modified \texttt{SQ} process in assembly. Please check the assembled code in \texttt{prog2-q3}
          \begin{verbatim}
SQ: 
    PUSH R12
    PUSH R13
    LD P1, R13
    AND R10, R13, R12
    BNZ R12, SQ_ODD
    SUB R12,R12,R12
    SUB R12,R10,R11
    BR SQ_END
SQ_ODD:
    MOV R10, R11
SQ_END:
    POP R12
    POP R13
    RET
    \end{verbatim}
\end{enumerate}

\end{document}