\documentclass{article}

% ------------------------------------ %
%             Document Info            %
% ------------------------------------ %


\usepackage[utf8]{inputenc}
\usepackage{amsmath, dsfont, mathtools, amssymb}
\usepackage{graphicx, float}
\usepackage[most]{tcolorbox}
\usepackage{enumitem}
\usepackage{hyperref}
\usepackage{multicol}
\usepackage{circuitikz}
\usetikzlibrary{calc}

\setlength{\parindent}{0em}
\setlength{\parskip}{0.5em}
\renewcommand{\baselinestretch}{1.25}

\usepackage[letterpaper, top=0.5in, left=0.5in, right=0.5in, bottom=1in]{geometry}

\begin{document}

% ------------------------------------ %
%                Content               %
% ------------------------------------ %

\section*{Question 1}
\hrule
\vspace{0.5cm}


\begin{enumerate}[label=\alph*.]
    \item (6 marks) Express the following statements using logical connectives and the propositions \( p \): "The user holds an account in the bank", \( q \): "The user can login to the cyberbanking portal", and \( r \): "The user enters the one-time password".
          \begin{enumerate}[label=\roman*.]
              \item "The user can login to the cyberbanking portal whenever the user holds an account in the bank and enters the one-time password".
              \item "If the user holds an account in the bank but has not entered the one-time password, then the user cannot login to the cyberbanking portal".
          \end{enumerate}
    \item (8 marks) Sherlock Holmes is in charge of finding who was involved in the theft of the crown jewels. The suspects are: Alan, Billy, Cooper, Donahue, and Ethan.
          \begin{itemize}
              \item Alan says that Billy or Cooper were involved, but not both.
              \item Billy says that Donahue or Alan were involved, but not both.
              \item Cooper says that Ethan or Billy were involved, but not both.
              \item Donahue says that Cooper or Ethan were involved, but not both.
              \item Ethan says that Cooper or Donahue were involved, but not both.
          \end{itemize}
          Of the five statements by the five suspects, Holmes has information that four are true, while one is false. Holmes has further been reliably informed by Inspector Lestrade that if Donahue was involved, then Cooper was involved. Help Holmes deduce which of the suspects were involved in the theft of the crown jewels, by using the methods of propositional logic.
\end{enumerate}

\section*{Question 2}
\hrule
\vspace{0.5cm}


\begin{enumerate}[label=\alph*.]
    \item (8 marks) Simplify the statements below, so that the negation appears only directly next to the predicates,
          if at all. The formula is then said to be in Negation Normal Form (NNF).
          \begin{enumerate}[label=\roman*.]
              \item $\neg(\forall x\forall y\left[(x<y)\to\exists z\left(y>z>x\right)\right])$
              \item $\lnot\forall x\exists!y\left[\lnot P(x,y)\land\lnot Q(x,y)\right]$
          \end{enumerate}
    \item (7 marks) In calculus, we say that a function \(f\) has the limit \(L\) as the argument \(x\) approaches \(a\), written as \(\lim_{x \to a} f(x) = L\), if and only if for every \(\epsilon > 0\) there exists a \(\delta > 0\) such that for all \(x\), the condition \((0 < |x − a| < \delta)\) implies \(| f (x) − L| < \epsilon\).
          \begin{enumerate}[label=\roman*.]
              \item Express the above idea of a limit in symbolic form, using predicate logic.
              \item Use your answer from the previous part to express \(\lim_{x \to a} f(x) \neq L\) in symbolic form.
              \item (7 marks) Prove that \( \forall x \{ \exists y [S(x, y) \land M(y)] \rightarrow \exists z [P(z) \land R(x, z)] \} \) logically implies \( [\neg \exists z P(z)] \rightarrow \forall x \forall y [S(x, y) \rightarrow \neg M(y)] \), where \( S(x, y) \), \( R(x, z) \), \( M(y) \), \( P(z) \) are generic predicates and the domains of all the variables are the same.
          \end{enumerate}
\end{enumerate}

\section*{Question 3}
\hrule
\vspace{0.5cm}


\begin{enumerate}[label=\alph*.]
    \item (8 marks) Show that \( (p \to r) \leftrightarrow (\neg r \to \neg p) \) is a Tautology by means of
          \begin{enumerate}[label=\roman*.]
              \item a Truth Table, and by using
              \item Logical Equivalence Rules.
          \end{enumerate}
    \item (8 marks) Let \( A, B, C \) be propositions. Is the compound proposition \( (B \to C) \to [\neg (A \to C) \land \neg (A \lor B)] \) logically equivalent to \( (B \land \neg C) \)? Explain your reasoning by means of
          \begin{enumerate}[label=\roman*.]
              \item a Truth Table, as well as by using
              \item Logical Equivalence rules.
          \end{enumerate}
\end{enumerate}



\section*{Question 4}
\hrule
\vspace{0.5cm}

\begin{enumerate}[label=\alph*.]
    \item (7 marks) Suppose that \( p, q, r, s, t, u \) are propositions such that \( p \to q, p \land t, \neg r \lor (\neg t \lor u) \) and \( q \to (r \land s) \) are all true. Can you determine the truth value of \( u \) with certainty? Explain your reasoning.
    \item (7 marks) Consider the following propositions:
          \begin{itemize}
              \item \( p \): If Joe studies, then he will score an A in the Discrete Math course.
              \item \( q \): If Joe `doesn’t play video games, then he will study.
              \item \( r \): Joe scores an A in the Discrete Math course.
              \item \( s \): Joe plays video games.
          \end{itemize}
          Determine whether the argument: \( (p \land q \land \neg r) \to s \) is valid. Explain your reasoning.
\end{enumerate}

\section*{Question 5}
\hrule
\vspace{0.5cm}

\begin{enumerate}[label=\alph*.]
    \item (6 marks) Prove that for any two integers \( x \) and \( y \), if \( xy \) and \( x + y \) are both even, then both \( x \) and \( y \) are necessarily even.
    \item (6 marks) The arithmetic mean of two positive real numbers \( x \) and \( y \) equals \( \frac{x + y}{2} \). The quadratic mean of two positive real numbers \( x \) and \( y \) equals \( \sqrt{\frac{x^2 + y^2}{2}} \). By computing the arithmetic and quadratic means of a few pairs of positive real numbers, formulate a conjecture about their relative sizes and prove your conjecture.
\end{enumerate}

\section*{Question 6}
\hrule
\vspace{0.5cm}

\begin{enumerate}[label=\alph*.]
    \item (7 marks) Alice observes that when she forms a natural number by simply writing down a single digit \( 3n \) times, the number always seems to be divisible by \( 3n \), for any integer \( n \geq 1 \). Help Alice prove that this is always true by using the Principle of Mathematical Induction.
    \item (7 marks) Are there any integer solutions to the equation \( x^2 - 36y = 6 \), i.e., are there any values \( (x, y) \in \mathbb{Z}^2 \) that satisfy the equation? Prove your answer.
    \item (8 marks) The terms of a sequence are given recursively as \( a_0 = 1, a_1 = 1 \) and \( a_n = 2 \cdot a_{n-1} + 3 \cdot a_{n-2} \) for \( n \geq 2, n \in \mathbb{Z} \). Prove using Strong Mathematical Induction that \( \frac12 \cdot 3^n + \frac12 \cdot (-1)^n \) is a closed form expression for this sequence, that is \( a_n = \frac12 \cdot 3^n + \frac12 \cdot (-1)^n \) for all \( n \in \mathbb{N} \).
\end{enumerate}

\end{document}