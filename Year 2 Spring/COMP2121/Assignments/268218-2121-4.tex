\documentclass{article}

% ------------------------------------ %
%             Document Info            %
% ------------------------------------ %

\usepackage{../../../LaTeX-Preamables/Assign}

\begin{document}
\newcommand{\documentcourse}{COMP2121}
\newcommand{\documentnumber}{4}

% ------------------------------------ %
%                Header                %
% ------------------------------------ %

\begin{minipage}{0.07\textwidth}
    \includegraphics[width=\linewidth]{../../../LaTeX-Preamables/LaTeX-Templates/HKULOGO256.png}
\end{minipage}
\hspace{0.02\textwidth}
\begin{minipage}{0.55\textwidth}
    \documentcourse

    Assignment \documentnumber

    SID: 3036268218
\end{minipage}
\begin{minipage}{0.35\textwidth}
    \begin{flushright}
        Jax

        \jobname.pdf

        \today
    \end{flushright}
\end{minipage}

\vspace{0.5cm}

\hrule

% ------------------------------------ %
%                Content               %
% ------------------------------------ %

\section*{Question 1}
\hrule
\vspace{0.5cm}

\begin{enumerate}
    \item Logic and Proofs [34 Marks]
          \begin{enumerate}
              \item (10 marks) Write the negation, converse, and contrapositive for each of the statements below in simplified form.
                    \begin{enumerate}[label=\roman*.]
                        \item $\forall x \in \mathbb{R}, x < 1 \rightarrow x^2 < 1$.
                        \item For every integer $x$ and every integer $y$, there is an integer $a$ such that if $x > 0$ then $a \cdot x > y$.
                    \end{enumerate}
              \item (10 marks) Find a proposition that is logically equivalent to:
                    \[
                        \neg[\neg p \rightarrow (q \rightarrow r)] \oplus \neg[\neg q \rightarrow (r \rightarrow p)] \oplus \neg[\neg r \rightarrow (p \rightarrow q)],
                    \]
                    and contains only the logical operations $\land$ and $\oplus$ (XOR).
              \item (14 marks) Let $n \in \mathbb{Z}^+$ and define a set of points on the plane as
                    \[
                        S := \{(x, y) \mid x \ge 0, y \ge 0, x + y < n\}.
                    \]
                    The points in $S$ are each colored red or green such that if $(x, y)$ is red, then $(x', y')$ is also colored red provided $x' \le x$ and $y' \le y$.
                    Let $C_1$ be the number of ways to choose $n$ green points such that all their $x$-coordinates are different. Let $C_2$ be the number of ways to choose $n$ green points such that all their $y$-coordinates are different. Prove using strong Mathematical Induction that $C_1 = C_2$ for all $n \in \mathbb{Z}^+$.
          \end{enumerate}
\end{enumerate}

\subsection*{Solutions to question 1}

\begin{enumerate}[label=(\alph*)]
    \item \begin{enumerate}[label=\roman*.]
              \item \textbf{Negation: } $\neg(\forall x \in \mathbb{R}, x < 1 \rightarrow x^2 < 1) \equiv \exists x \in \mathbb{R}, x < 1 \land x^2 \ge 1$.\\
                    \textbf{Converse: } $\forall x \in \mathbb{R}, (x^2 < 1 \rightarrow x < 1)$.\\
                    \textbf{Contrapositive: } $\forall x \in \mathbb{R}, (x^2 \ge 1 \rightarrow x \ge 1)$
              \item \textbf{Negation: } $\exists x, y \in \mathbb{Z}, \forall a \in \mathbb{Z}, (x > 0) \rightarrow (ax \leq y)$\\
                    \textbf{Converse: } $\forall x, y \in \mathbb{Z}, \exists a \in \mathbb{Z}, (ax > y) \rightarrow (x > 0)$\\
                    \textbf{Contrapositive: } $\forall x, y \in \mathbb{Z}, \exists a \in \mathbb{Z}, (ax \leq y) \rightarrow (x \leq 0)$\\
          \end{enumerate}
    \item \begin{align*}
               & \neg[\neg p \rightarrow (q \rightarrow r)] \oplus \neg[\neg q \rightarrow (r \rightarrow p)] \oplus \neg[\neg r \rightarrow (p \rightarrow q)]                                     \\
               & \neg[ p \lor (q \rightarrow r)] \oplus \neg[ q \lor (r \rightarrow p)] \oplus \neg[ r \lor (p \rightarrow q)]                                                                      \\
               & \neg[ p \lor (\neg q \lor r)] \oplus \neg[ q \lor (\neg r \lor p)] \oplus \neg[ r \lor (\neg p \lor q)]                                                                            \\
               & \neg p \land \neg (\neg q \lor r) \oplus  \neg q \land \neg (\neg r \lor p) \oplus  \neg r \land \neg (\neg p \lor q)                                                              \\
               & [\neg p \land q \land \neg r] \oplus  [\neg q \land r \land \neg p] \oplus  [\neg r \land p \land \neg q]                                                                          \\
               & [(1 \oplus p) \land q \land (1 \oplus r)] \oplus  [(1 \oplus q) \land r \land (1 \oplus p)] \oplus  [(1 \oplus r) \land p \land (1 \oplus q)]  & \because A \oplus 1 \equiv \neg A \\
          \end{align*}
    \item Let $x_i$ be the number of green points with a $x$ coordinate of $i$, and $y_i$ be the number of green points with $y$ coordinate of $i$. Our goal is to show that $x_0\times x_1 \times \dots x_{n-1} = y_0\times y_1 \times \dots y_{n-1}$. (i.e. $C_1 = C_2$)

          Consider base case $n=1$, we can easily see that $x_0=y_0=\begin{cases}1 & if\ x_0=1\\0 & otherwise\end{cases}$.

          Inductive step: Assume that $\forall k \leq n$ is true, consider $k=n+1$:

          Consider the $S=\{(x,y)\mid x\geq0,y\geq0,x+y<n+1\}$. Observe that points on line $x+y=n$ is either red or green.

          \begin{enumerate}
              \item \textbf{Case 1: } All points are green on the line $x+y=n$. By induction hypothesis, $\{x_0, x_1, \dots, x_{n-1}\} = \{y_0, y_1, \dots, y_{n-1}\}$. As all points on line $x+y=n$ is green, we can add 1 to each $x_i$ and $y_i$ to get $\{x_0, x_1, \dots, x_{n}\} = \{y_0, y_1, \dots, y_{n}\}$.
              \item \textbf{Case 2: } At least one point on the line $x+y=n$ is red. Suppose
                    (a, n-a) is red. This splits the grid into two subgrids (isosceles right-angled triangles): one with $x>a$ and $y<n-a$, and another with $y>n-a$ and $x<a+1$. By induction, each subgrid satisfies $C_1 = C_2$. The total $C_1$ and $C_2$ for the entire grid are products of the respective counts from each subgrid, which are equal by induction.
          \end{enumerate}

          Therefore, $C_1 = C_2$ for all $n \in \mathbb{Z}^+$.


\end{enumerate}

\newpage
\section*{Question 2}
\hrule
\vspace{0.5cm}

\begin{enumerate}
    \item Sets, Relations and Functions [32 Marks]
          \begin{enumerate}[label=(\alph*)]
              \item (10 marks) Let $A$ and $B$ be arbitrary finite sets. Prove that $A \cup B = (A \setminus (A \cap B)) \cup B$, using well-known Set Identities.
              \item (12 marks) In this question, we are going to consider equivalence relations defined on the set $S := \{u, v, w, x, y, z\}$.
                    \begin{enumerate}[label=\roman*.]
                        \item How many of the equivalence relations on $S$ have exactly two equivalence classes of size three?
                        \item How many of the equivalence relations on $S$ have exactly one equivalence class of size three?
                        \item How many of the equivalence relations on $S$ have at least one equivalence class with three or more elements?
                    \end{enumerate}
              \item (10 marks) Compare the asymptotic behaviour of the following pair of functions:
                    \[
                        f(n) = n^{\log n} (\log \sqrt{n})^{\sqrt{n}},
                        \quad
                        g(n) = (\log n)^n.
                    \]
                    That is, state and explain whether each of the following holds: (i) $f(n) = O(g(n))$, (ii) $g(n) = O(f(n))$.
          \end{enumerate}
\end{enumerate}

\subsection*{Solutions to question 2}

\begin{enumerate}[label=(\alph*)]
    \item \begin{align*}
              (A \setminus (A \cap B))                   & \text{ respresents the set } \{a | a \in A \land a \notin (A \cap B)\}                           \\
              \text{Notice } \overline{(A \cap B)}       & = \{a | a \in U \land a \notin (A \cap B)\}                                                      \\
              \therefore A \cap \overline{(A \cap B)}    & = \{a | a \in A \land a \notin (A \cap B)\}                                                      \\
                                                         & = (A \setminus (A \cap B))                                                                       \\
              \therefore (A \setminus (A \cap B)) \cup B & = (A \cap \overline{(A \cap B)}) \cup B                                                          \\
                                                         & = (A \cap (\overline{A} \cup \overline{B})) \cup B                     & \text{De Morgan's law}  \\
                                                         & = (A \cap \overline{A}) \cup (A \cap \overline{B}) \cup B              & \text{Distributive law} \\
                                                         & = \emptyset \cup (A \cap \overline{B}) \cup B                          & \text{Complement law}   \\
                                                         & = (A \cap \overline{B}) \cup B                                         & \text{Identity law}     \\
                                                         & (A \cup B) \cap (\overline{B} \cup B)                                  & \text{Distributive law} \\
                                                         & = (A \cup B) \cap U                                                    & \text{Complement law}   \\
                                                         & = A \cup B                                                             & \text{Identity law}     \\
          \end{align*}
    \item \begin{enumerate}[label=\roman*.]
              \item $\forall x, y \in S$, if $(x,y)$ is in an equiv. relation with one another, they will belong in either partition or equiv. classes $A$ or $B$. This is because $ [x] = [y] \text{ iff } x \sim y,$. Therefore, we can simply consider the paritions of the number in $S$ to figure out the answer.\\
                    Number of ways pick 3 elements out of 6 (and have the rest be a set) is $\binom{6}{3} = 20$. However, $A$ and $B$ are interchangeable, so we need to divide by 2, which means the answer is $\frac{1}{2} \binom{6}{3} = 10$.
              \item To partition the numbers so that 3 is in a group and the rest is not:
                    \begin{itemize}
                        \item Rest of the 3 numbers in a set each : $1$ way
                        \item 2 numbers in a set and 1 number in a set: $\binom{3}{2}= 3$ ways
                    \end{itemize}
                    Total ways to handle rest of the numbers: $1 + 3 = 4$ ways\\
                    Total ways: $4 \times \binom{6}{3} = 4 \times 20 = 80$ ways

              \item \begin{itemize}
                        \item Number of ways with at least one class of size 3: $80+10 = 90$ ways
                        \item Number of ways with one class of size 4: $\binom{6}{4} \times (1 + 1) = 30$ ways
                        \item Number of ways with one class of size 5: $\binom{6}{5} \times (1) = 6$ ways
                        \item Number of ways with one class of size 6: $1$ way
                        \item Total ways: $90 + 30 + 6 + 1 = 127$ ways
                    \end{itemize}

          \end{enumerate}
    \item
          \begin{enumerate}[label=\roman*.]
              \item To compare, we take the limit of the ratio of the functions to infinite: \begin{align*}
                                          & \lim_{n\to\infty}\frac{f(n)}{g(n)}                                                                                                 \\
                                          & = \lim_{n\to\infty}\frac{n^{\log n} (\log \sqrt{n})^{\sqrt{n}}}{(\log n)^n}                                                        \\
                                          & = \lim_{n\to\infty}\frac{n^{\log n} {\frac12}^{\sqrt{n}} (\log n)^{\sqrt{n}}}{(\log n)^n}                                          \\
                                          & = \lim_{n\to\infty}{n^{\log n} {\frac12}^{\sqrt{n}} (\log n)^{\sqrt{n}-n}}                                                         \\
                        \text{Consider: } & \lim_{n\to\infty}\log(n^{\log n} {\frac12}^{\sqrt{n}} (\log n)^{\sqrt{n}-n})                                                       \\
                                          & = \lim_{n\to\infty}\log n\log n + \sqrt{n}\log \frac12 + (\sqrt{n}-n)\log \log n                                                   \\
                                          & = \lim_{n\to\infty}\log n\log n + \sqrt{n}\log \frac12 + \sqrt{n}\log \log n - n\log \log n                                        \\
                                          & = \lim_{n\to\infty}n (\frac{1}{n} \log n\log n + \frac{1}{n} \sqrt{n}\log \frac12 + \frac{1}{n} \sqrt{n}\log \log n - \log \log n) \\
                                          & = \lim_{n\to\infty} -n \log \log n                                                                                                 \\
                                          & = -\infty                                                                                                                          \\
                        \therefore        & \lim_{n\to\infty}\frac{f(n)}{g(n)} \to 10^{-\infty}                                                                                \\
                                          & = 0
                    \end{align*}

                    Therefore, the growth rate of $g(n)$ is faster than $f(n)$, and hence $f(n) = O(g(n))$.

              \item $f(n) = O(g(n)) \land f(n) << g(n)\implies g(n) \neq O(f(n))$.
          \end{enumerate}
\end{enumerate}

\newpage
\section*{Question 3}
\hrule
\vspace{0.5cm}

\begin{enumerate}
    \item Counting and Probability [34 Marks]
          \begin{enumerate}[label=(\alph*)]
              \item (10 marks) In a certain lottery, six numbers are drawn out of \( S = \{1, 2, \ldots, 49\} \) but the numbers drawn are put back into the pool right after being selected. In order to win the first prize jackpot, Tom's ticket must have the same multiset of numbers as the one drawn, regardless of the order in which the numbers were drawn. How many lottery tickets must Tom buy in order to make sure that he wins the jackpot in this lottery?
              \item (12 marks) Consider a \(3 \times 3\) matrix \(M\). Amy chooses each entry \(m_{i,j}\) for \(1 \le i, j \le 3\) of the matrix from \(\{0, 1\}\) in a uniformly random manner, i.e., each entry is set to 0 or 1 with probability \(\frac{1}{2}\). What is the probability that Amy ends up with a matrix \(M\) that does not have a constant \(2 \times 2\) sub-matrix? Note that a constant sub-matrix is one in which all entries are the same.
              \item (12 marks) Let \(n \in \mathbb{Z}^+\) be arbitrary. For \(k \in \mathbb{Z}^+\), a sequence \((a_1, a_2, \ldots, a_k)\) of integers satisfying \(a_i \ge 0\) for all \(i \in \{1, 2, \ldots, k\}\), and with \(\sum_{i=1}^{k} a_i = n\) is termed a weak composition of \(n\) into \(k\) parts. How many weak compositions of 10 into five parts are there such that exactly two parts are 0?
          \end{enumerate}
\end{enumerate}

\subsection*{Solutions to question 3}

\begin{enumerate}[label=(\alph*)]
    \item The count of possible lottery numbers is $\binom{49+6-1}{6} = \binom{54}{6} = 25827165$ (each number can repeat, assuming the all possible lottery numbers can be purchased)
    \item Focusing on only \textit{continous} sub-matrices, there are 4 possible continous submatrices (TL, TR, BL, BR 2x2). Let $A_i$ be the event that submatrix $i$ is a constant submatrix.

          The total number of posible matrices is $2^9 = 512$. Consider the following probabilities

          \begin{enumerate}
              \item Single constant matrix:
                    \begin{enumerate}
                        \item Probability for a matrix to be all 0 or 1 is $(\frac12)^4 = \frac{1}{16}$
                        \item Probability for a matrix to be constant is hence $P(A_i) = \frac{1}{16} \times 2 = \frac{1}{8}$
                        \item $\sum_{i=1}^{4} P(A_i) = 4\times\frac{1}{8} = \frac{1}{2}$
                    \end{enumerate}
              \item Double constant matrices:
                    \begin{enumerate}
                        \item Probability for existence of two constant matrices diagonally is $(\frac12)^7 \times 2 = \frac{1}{64}$
                        \item Probability for existence of two constant matrices vertically/horizontally is $(\frac12)^6 \times 2 = \frac{1}{32}$
                        \item $\sum_{i=1}^{4} P(A_i \cap A_j) = 2 \times{1}{64} + 4 \times {1}{32} = \frac{5}{32}$
                    \end{enumerate}
              \item Triple constant matrices: $\sum_{i=1}^{4} P(A_i \cap A_j \cap A_k) = 4 \times 2 \times (\frac12)^8 = \frac{1}{32}$
              \item Entire constant matrix: $\sum_{i=1}^{4} P(A_i \cap A_j \cap A_k \cap A_l) = 2 \times (\frac12)^9 = \frac{1}{256}$
          \end{enumerate}

          Therefore, by inclusion exclusion principle:

          \begin{align*}
              P(A_1 \cup A_2 \cup A_3 \cup A_4)          & = \frac12 - \frac{5}{32} + \frac{1}{32} - \frac{1}{256} = \frac{95}{256} \\
              P(\text{No constant continous sub-matrix}) & = 1 - \frac{95}{256} = \frac{161}{256}
          \end{align*}

    \item Select two positions to be 0: $\binom{5}{2}$, Select 3 numbers to add to 10 where $a \geq 1$: $\binom{10-1}{3-1}$. Solution to question: $\binom{5}{2} \times \binom{9}{2} = 360$
\end{enumerate}

\end{document}
