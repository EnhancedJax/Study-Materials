\documentclass{article}

% ------------------------------------ %
%             Document Info            %
% ------------------------------------ %

\usepackage{../../LaTeX-Preamables/Clean}
\newcommand{\documentdate}{Fall 2025}
\newcommand{\documenttype}{Notes for HKU}

% ------------------------------------ %
%              Title page              %
% ------------------------------------ %
\begin{document}
\begin{titlepage}
    \null\vfill % Add vertical space to center the title and author

    \centering
    \Huge\textbf{\documentname}

    \vspace{0.1cm}
    \Large\textbf{\documenttype\ $\cdot$ \documentdate}

    \vspace{1cm}
    \normalsize\textbf{Author:} \documentauthor

    \normalsize\textbf{Contact:} \documentauthorcontact

    MORE notes on my \href{https://enhancedjax.github.io/#/hkunotes}{website}!
    \vfill % Add vertical space to center the remaining space
    \textcolor{gray}{Made for personal use only. Unmodified re-distribution is allowed. Content for reference only.}
\end{titlepage}

% ------------------------------------ %
%                  TOC                 %
% ------------------------------------ %

\newpage
\tableofcontents
\newpage

% ------------------------------------ %
%               Document               %
% ------------------------------------ %

\begin{definition}
    {General definition of asymptotic notations}
    For non-strict boundaries:
    \[T(n) = A(g(n))\ \text{iff}\ \mathcolorbox{yellow}{\exists\ c > 0}:\]
    \[T(n) \simeq c\cdot g(n)\ \forall\ n \geq n_0 > 0\]
    For strict boundaries:
    \[T(n) = a(g(n))\ \text{if}\ \mathcolorbox{yellow}{\forall\ c > 0\ \exists\ n_0 \geq 0}:\]
    \[T(n) \simeq c\cdot g(n)\ \forall\ n \geq n_0\]
\end{definition}

\begin{definition}
    {Asymptotic notations / growth rate}
    \begin{tabular}{|c|c|c|c|}
        \hline
        \textbf{Notation} & \textbf{Condition}                                                                         & \textbf{Asymptotic boundary} & \textbf{Name}               \\ \hline
        $O(g)$            & $T(n) \mathcolorbox{yellow}{\leq} c\cdot g(n)$                                             & Upper                        & Big O                       \\
        $\Omega(g)$       & $T(n) \mathcolorbox{yellow}{\geq} c\cdot g(n)$                                             & Lower                        & Big Omega                   \\
        $\Theta(g)$       & $ c_1\cdot g(n)\mathcolorbox{yellow}{\leq} T(n) \mathcolorbox{yellow}{\leq} c_2\cdot g(n)$ & Tight                        & Big Theta / Order of growth \\
        $o(g)$            & $T(n) \mathcolorbox{yellow}{<} c\cdot g(n)$                                                & Strictly upper               & Little o                    \\
        $\omega(g)$       & $T(n) \mathcolorbox{yellow}{>} c\cdot g(n)$                                                & Strictly lower               & Little omega                \\ \hline
    \end{tabular}
    We can easily identify the growth rate of $T(n)$ by looking for the highest order of growth term in the expression, then checking for the condition.
\end{definition}

\begin{example}
    {Disproving Big O notation}

    Consider $T(n) = 3n^3 + 1$, to show that $T(n) \neq O(n^2)$:
    \begin{align*}
        3n^3 + 1           & \leq c \cdot n^2 & \forall\ n \geq n_0 \\
        3n^3 + 1           & \leq c\cdot n^2  & \forall\ n \geq 2   \\
        3n + \frac{1}{n^2} & \leq c           & \forall\ n \geq 2   \\
    \end{align*}
    As this expression cannot hold true for all $n \geq 2$ for a specific $c$ value, we can conclude that $T(n) \neq O(n^2)$.

    (Example: if $c = 10$ the equation does not hold when let's say $n = 100$)
\end{example}

\begin{example}
    {Proving Little o notation}

    To show that $T(n) \in o(n^4)$:
    \begin{align*}
        3n^3 + 1                    & < c \cdot n^4 & \forall\ n \geq n_0 \\
        \frac{3}{n} + \frac{1}{n^4} & < c           & \forall\ n \geq n_0 \\
    \end{align*}
    As this express can hold true for any $c > 0$ with sufficiently large $n_0$, we can conclude that $T(n) \in o(n^4)$.

    (Example: if $c = 1$ the equation holds with $n_0 = 100$)
\end{example}



\end{document}