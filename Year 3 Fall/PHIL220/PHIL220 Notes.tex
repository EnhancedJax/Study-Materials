\documentclass{article}

% ------------------------------------ %
%             Document Info            %
% ------------------------------------ %

\usepackage{../../LaTeX-Preamables/CleanV2}
\newcommand{\documentdate}{Fall 2025}
\newcommand{\documenttype}{Notes for UBC}

% ------------------------------------ %
%              Title page              %
% ------------------------------------ %
\begin{document}
\begin{titlepage}
    \null\vfill % Add vertical space to center the title and author

    \centering
    \Huge\textbf{\documentname}

    \vspace{0.1cm}
    \Large\textbf{\documenttype\ $\cdot$ \documentdate}

    \vspace{1cm}
    \normalsize\textbf{Author:} \documentauthor

    \normalsize\textbf{Contact:} \documentauthorcontact

    MORE notes on my \href{https://jaxtam.dev/notes}{website}!
    \vfill % Add vertical space to center the remaining space
    \textcolor{gray}{Made for personal use only. Unmodified re-distribution is allowed. Content for reference only.}
\end{titlepage}

% ------------------------------------ %
%                  TOC                 %
% ------------------------------------ %

\newpage
\hypertarget{toc}{}
\tableofcontents

\newpage

% ------------------------------------ %
%               Document               %
% ------------------------------------ %

\section{Introduction}

\begin{definition}{Statement (Sentence)}
    Strings that can be classified as true or false.

    \tcblower

    \textbf{Examples:}
    \begin{itemize}
        \item The sun will rise tomorrow. $\checkmark$
        \item 2 + 2 = 5 $\checkmark$
        \item Where is the library? $\times$
        \item Please close the door. $\times$
    \end{itemize}
\end{definition}

\begin{definition}
    {Argument}
    A set of statements related in such a way that the \textbf{conclusion} statement is supported by the other \textbf{truthful} statements (\textbf{premises}).

    Each argument has one conclusion and one or more premises.
\end{definition}

\begin{theorem}{Standard form}

    \textit{Example argument in non-standard form:} "Marijuana should not be legalized because it is potentially dangerous and not enough is known about its long-term effects, and because use of marijuana leads to use of hard drugs."

    The standard form would be as follows:

    \begin{itemize}
        \item \textbf{(1)} Marijuana is potentially dangerous
        \item \textbf{(2)} Not enough is known about its long-term effects.
        \item \textbf{(3)} Marijuana leads to use of hard drugs.
        \item \textbf{(C)} Marijuana should not be legalized.
    \end{itemize}
\end{theorem}

\begin{knBox}
    {Indicator words}
    Words that indicate the presence of premises or conclusions in an argument.

    \textbf{Premise indicators:} since, because, owing to, given that, seeing that, for the reason that, etc.

    \textbf{Conclusion indicators:} therefore, we may infer, we may conclude that, it must be that, entails that, so, hence, thus, implies that, consequently as a result, etc.
\end{knBox}

\begin{knBox}
    {Argument analysis}

    Steps:
    \begin{enumerate}
        \item Reconstruct the argument in standard form.
        \item Test: Do premises support conclusion?
        \item Test: Are each of the premises true?
    \end{enumerate}
\end{knBox}

\subsection{Logical notions}


\begin{definition}
    {Joint possibility}

    Statements are \textbf{jointly possible} \textit{iff} they can all be true at the same time.

    \tcblower

    \begin{itemize}
        \item \textit{It is raining here. It is not raining here.} $\times$
        \item \textit{It is raining in Vancouver. It is not raining in Hong Kong.} $\checkmark$
    \end{itemize}

\end{definition}

\begin{definition}
    {Necessary truth / falsehood}

    A statement that is true / false in all possible circumstances.
\end{definition}

\begin{definition}
    {Contingency}

    A statement that is true in some possible circumstances and false in others.
\end{definition}

\begin{definition}
    {Necessary equivalence}

    Statements that are both true or both false in all possible circumstances.

    \tcblower

    \textit{"I don't want apple or bananas"} \& \textit{"I don't want apple and don't want bananas"}
\end{definition}

\subsection{Validity and Soundness}

\begin{knBox}
    {Consequence}

    A statement is a \textbf{consequence} of (follows / entails) a set of premises \textit{iff} it is \textbf{impossible} for the premises to be all true and the conclusion false.
\end{knBox}

\begin{definition}
    {Validity}

    An argument is \textbf{valid / deductive} \textit{iff} the conclusion is a \textbf{consequence} of the premises.

\end{definition}

\begin{definition}
    {Soundness}

    An argument is \textbf{sound} \textit{iff} it is valid and all its premises are true.
\end{definition}

\begin{itemize}
    \item Valid arguments can have false conclusions.
    \item \textbf{Necessarily true} conclusions $\implies$ valid argument.
    \item Jointly inconsistent premises $\implies$ valid, unsound argument.
\end{itemize}

\section{Truth-functional logic}

Ordinary languages are full of vagueness and ambiguity, meaning can vary, in contrast to formal languages which are precise and unambiguous.

\begin{knBox}
    {Atomic letters}

    Atomic letters are used to represent simple statements: $A_1, Z_{123}$ etc.
\end{knBox}

\begin{theorem}{Propositions}
    A proposition $p$ is a statement that is either true or false, but not both, which must \textbf{declare a fact} and must have a truth value.

    \tcblower
    e.g. ``It is raining'', ``2 + 2 = 4'', ``The sky is blue''
\end{theorem}

\begin{definition}{Definitions}
    \begin{tabular}{|c|c|c|c|c|}
        \hline
        Precedence & Definition      & Symbol                & Meaning                       & Truth table TLRF \\
        \hline
        1          & Negation        & $\neg p$              & Not $p$                       & -                \\
        2          & Conjunction     & $p \land q$           & $p$ and $q$                   & TFFF             \\
        3          & Disjunction     & $p \lor q$            & $p$ or $q$                    & TTTF             \\
        4          & Exclusive or    & $p \oplus q$          & $p$ or $q$ but not both (XOR) & FTTF             \\
        5          & Implication     & $p \rightarrow q$     & If $p$ then $q$               & TFTT             \\
        6          & Bi-conditionals & $p \leftrightarrow q$ & $p$ if and only if $q$ (XNOR) & TFFT             \\
        \hline
    \end{tabular}

    \textit{Precedence:} $p \rightarrow q \land r \lor s \equiv p\rightarrow ((q \land r)\lor s)$

    For an implication $p \rightarrow q$:
    \begin{itemize}
        \item \textbf{Converse}: $q \rightarrow p$
        \item \textbf{Inverse}: $\neg p \rightarrow \neg q$
        \item \textbf{Contrapositive}: $\neg q \rightarrow \neg p$ (Same truth value as the original)
    \end{itemize}

    \tcblower
    \textit{Why implication $|..T.|$?} The statement can still hold true even if $F \rightarrow T$:

    Consider: ``Bob goes to work, it's Monday.'' The day can still be Monday even if Bob doesn't go to work.
\end{definition}

\subsection{Logic Equivalences}

\begin{theorem}{Tautology}
    Proposition that is always true, regardless of the truth values of the propositions it contains.

    \tcblower
    \begin{itemize}
        \item $p \lor \neg p$ is always true
        \item $p \rightarrow p$ is always true
    \end{itemize}
\end{theorem}

\begin{theorem}{Contradiction}
    Proposition that is always false.

    \tcblower
    \begin{itemize}
        \item $p \land \neg p$ is always false
    \end{itemize}
\end{theorem}

\begin{theorem}{Logical Equivalence}
    If $p \leftrightarrow q$ is a tautology, then $p$ and $q$ are logically equivalent, denoted as $p \equiv q$.

    \tcblower
    \begin{itemize}
        \item $p \land q \equiv q \land p$
        \item $p \lor q \equiv q \lor p$
    \end{itemize}
\end{theorem}

% The following are the methods \& laws of propositional logic:

% \begin{definition}{Laws}
%     \begin{tabular}{|c|c|}
%         \hline
%         Category        & Rule                                                                   \\
%         \hline
%         Identity        & $p \land \mathbf{T} \equiv p$                                          \\ & $p \lor \mathbf{F} \equiv p$ \\
%         Domination      & $p \lor \mathbf{T} \equiv \mathbf{T}$                                  \\ & $p \land \mathbf{F} \equiv \mathbf{F}$ \\
%         Idempotent      & $p \lor p \equiv p$                                                    \\ & $p \land p \equiv p$ \\
%         Negation        & $p \lor \neg p \equiv \mathbf{T}$                                      \\ & $p \land \neg p \equiv \mathbf{F}$ \\
%         Commutative     & $p \lor q \equiv q \lor p$                                             \\ & $p \land q \equiv q \land p$ \\
%         Associative     & $(p \lor q) \lor r \equiv p \lor (q \lor r)$                           \\ & $(p \land q) \land r \equiv p \land (q \land r)$ \\
%         Double Negation & $\neg \neg p \equiv p$                                                 \\
%         Bi-Implication  & $p \leftrightarrow q \equiv (p \rightarrow q) \land (q \rightarrow p)$ \\
%         Contrapositive  & $p \rightarrow q \equiv \neg q \rightarrow \neg p$                     \\
%         Implication     & $p \rightarrow q \equiv \neg p \lor q$                                 \\
%         Distributive    & $p \land (q \lor r) \equiv (p \land q) \lor (p \land r)$               \\ & $p \lor (q \land r) \equiv (p \lor q) \land (p \lor r)$ \\
%         De Morgan's     & $\neg (p \land q) \equiv \neg p \lor \neg q$                           \\ & $\neg (p \lor q) \equiv \neg p \land \neg q$ \\
%         \hline
%     \end{tabular}
% \end{definition}

\subsection{Satisfiability}

\begin{theorem}{Satisfiability}
    Proposition can be made true by some sort of assignment of truth values to its variables, called the \textbf{solution}.

    \tcblower
    \begin{itemize}
        \item $p \land q$ is satisfiable when $p = T, q = T$
        \item $p \land \neg p$ is unsatisfiable
    \end{itemize}
\end{theorem}

\section{Natural deduction}

Natural deduction is a proof system that uses a set of inference rules to derive conclusions from premises in a step-by-step manner.

\begin{knBox}
    {Formatting proofs}
    \begin{itemize}
        \item Each line should be numbered
        \item Each line contains a statement and a justification by the rules with a number (n) or range (r) of lines
        \item Premises and assumptions are underlined
        \item \textbf{Subproofs} are proofs that start with an assumption and end when the assumption is discharged
        \item Each subproof starting from the assumption should be indented
        \item Use vertical bars to indicate the level of indentation
    \end{itemize}
\end{knBox}

\begin{theorem}
    {Contradiction}

    $\bot$ denotes a contradiction, which is a statement that is always false. Can be introduced by \textbf{negation elimination} and eliminated by \textbf{negation introduction}.
\end{theorem}

\begin{definition}
    {Natural deduction rules}

    \colorbox{pink}{Start subproof}, \colorbox{yellow}{discharge assumption}, \colorbox{orange}{discharge multiple assumptions}

    \begin{tabular}{l|lp{8cm}}
        \hline
        \textbf{Notation}           & \textbf{Name}                                 & \textbf{Explanation}                                                                                                    \\
        \hline
        R, n                        & Reiteration                                   & Copying a previous line                                                                                                 \\
        \underline{PR}              & Premise                                       & Using a premise from the argument                                                                                       \\
        \underline{AS}              & \colorbox{pink}{Assumption}                   & Start subproof by assuming a statement                                                                                  \\
        $\land I$, n1, n2           & Conjunction Introduction                      & From $p$ and $q$, infer $p \land q$                                                                                     \\
        $\land E$, n                & Conjunction Elimination                       & From $p \land q$, infer $p$ or $q$                                                                                      \\
        $\rightarrow I$, r          & \colorbox{yellow}{Implication Introduction}   & \textit{Assuming} $p$ leading to $q$, infer $p \rightarrow q$                                                           \\
        $\rightarrow E$, n1, n2     & Implication Elimination                       & From $p$ and $p \rightarrow q$, infer $q$                                                                               \\
        $\leftrightarrow I$, r1, r2 & \colorbox{orange}{Biconditional Introduction} & \textit{Assuming} $p$ leading to $q$ and \textit{assuming} $q$ leading to $p$, infer $p \leftrightarrow q$              \\
        $\leftrightarrow E$, n1, n2 & Biconditional Elimination                     & From $p \leftrightarrow q$ and $p$ or $q$ infer $q$ or $p$                                                              \\
        $\lor I$, n                 & Disjunction Introduction                      & From $p$, infer $p \lor q$ or from $q$, infer $p \lor q$                                                                \\
        $\lor E$, n1, r1, r2        & \colorbox{orange}{Disjunction Elimination}    & From $p \lor q$, and subproofs \textit{assuming} $p$ leading to $r$ and \textit{assuming} $q$ leading to $r$, infer $r$ \\
        $\neg I$, r                 & \colorbox{yellow}{Negation Introduction}      & \textit{Assuming} $p$ leading to $\bot$, infer $\neg p$                                                                 \\
        $\neg E$, n1, n2            & Negation Elimination                          & From $p$ and $\neg p$, infer $\bot$                                                                                     \\
        IP r                        & \colorbox{yellow}{Indirect Proof}             & \textit{Assuming} $\neg p$ leading to $\bot$, infer $p$  (Proof by Contradiction)                                       \\
        X, n                        & Explosion                                     & From $\bot$, infer any statement $p$                                                                                    \\
    \end{tabular}
\end{definition}


\end{document}