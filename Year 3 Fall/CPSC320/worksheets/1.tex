\documentclass{article}
\usepackage{../../../LaTeX-Preamables/Base}

\begin{document}

Imagine for example the task faced by UBC’s co-op office each semester, which seeks to match hundreds of student applicants to employer internships. To keep the problem as simple as possible for now, assume that every applicant has a full ranking of employers and vice versa (no ties).

\textbf{Full ranking: } Every student ranks every company, and every company ranks every student.

\subsection*{Notation}

\begin{itemize}
    \item $s$: students, $c$: companies
    \item $n_s = n_c = n$ number of students, number of companies
    \item Student preferences array: $P_s[s_i]=[c_1, c_2, c_3, \ldots]$
    \item Company preferences array: $P_c[c_i]=[s_1, s_2, s_3, \ldots]$
\end{itemize}

\subsection*{Notation for describing solution}

Return mapping $s \to c$

\subsection*{What makes a solution good}

\begin{itemize}
    \item Return \textbf{stable matching}: No student-company pair $(s_i, c_j)$ such that $s_i$ prefers $c_j$ over their assigned company and $c_j$ prefers $s_i$ over their assigned student.
    \item Return a \textbf{perfect matching}: Every student is assigned a company and every company is assigned a student.
    \item Respects the preferences of both sides as much as possible
    \item Deterministic results
    \item Efficient algorithm
\end{itemize}

\end{document}