\section{Recursion}

Solves a problem by breaking it down into smaller instances of the same problem. If you can solve a smaller instance of the problem, you can solve a larger one.

\begin{enumerate}
    \item \textbf{Base Case(s)}: The simplest instance(s) of the problem that can be solved directly without further recursion.
    \item \textbf{Recursive Case(s)}: More complex instances of the problem that are solved by breaking them down and making recursive calls.
\end{enumerate}

\begin{knBox}
    {Tower of Hanoi}
    \textbf{Goal: }Move $n$ disks from peg $A \implies C$ using peg $B$.

    \textbf{Rules: }Only one disk can be moved at a time, and a disk can only be placed on top of a larger disk.

    We define the problem as $H(num, from, to, via)$:
    \begin{enumerate}
        \item Move top $n-1$ disks $A \implies B$ via $C$: $H(n-1, A, B, C)$
        \item Move 1 disk $A \to C$
        \item Move top $n-1$ disks $B \implies C$ via $A$: $H(n-1, B, C, A)$
    \end{enumerate}
\end{knBox}

\subsection{Recurrence relation}

\begin{definition}
    {Recurrence relation / equation}
    A mathematical equation which is defined in terms of itself.
    \[f(n)=\begin{cases}\quad1&n=1\\2f(n-1)+1&n>1\end{cases}\]
\end{definition}

\begin{theorem}
    {Solve a recurrence relation - subsitution method}
    We "solve" a recurrence relation by finding a closed-form expression for $f(n)$ in terms of $n$.
    \begin{enumerate}
        \item Expand the relation as $f(n)$ to 3 levels
        \item Observe the pattern of $f(n)$
        \item Generalize the pattern in terms of the level $k$
        \item Using the base case to define $k$ in terms of $n$
        \item Subsitute $k$ back into the generalization
    \end{enumerate}
\end{theorem}

\begin{knBox}
    {Relevant mathematics}
    \begin{enumerate}
        \item Geometric series: $S_n = a(1 + r + r^2 + \cdots + r^{n-1}) = a\frac{r^n - 1}{r-1}$
        \item Arithmetic series: $S_n = n(\frac{a_1+a_n}{2}),\quad n=\text{number of terms}, a=\text{first / last term}$
        \item Factorial: $n! = n \times (n-1) \times (n-2) \times \cdots \times 1$
        \item Falling factorial: $n^{\underline{k}} = n(n-1)(n-2)\cdots(n-k+1)=\frac{n!}{(n-k)!}$
        \item Logarithm: $a^k=b\implies k=\log_a b$
        \item Logarithm property: $k^{\log a} = a^{\log k}$
    \end{enumerate}
\end{knBox}

\begin{theorem}
    {The master theorem}
    The master theorem can give us the order of growth (Big Theta) of a recurrence relation.

    For $T(n) = a\cdot T(\frac{n}{b}) + f(n),\quad T(1) = c,\quad a, c > 0,\ b > 1,\ f(n)\in \Theta(n^k),\ d\geq 0$, we have:
    \[T(n) = \begin{cases}
            \Theta(n^{\log_b a}) & \text{if } a > b^k \\
            \Theta(n^k \log n)   & \text{if } a = b^k \\
            \Theta(n^k)          & \text{if } a < b^k
        \end{cases}\]
\end{theorem}

\subsection{Mathematical induction}

\begin{definition}
    {Mathematical induction}
    A method of mathematical proof that proves a statement for all natural numbers.
    \begin{enumerate}
        \item \textbf{Base case}: Prove the statement for the first natural number in the statement's range.
        \item \textbf{Inductive step}: Assume the statement is true for $n$, then prove it is also true for $n+1$.
    \end{enumerate}
\end{definition}